%% -!TEX root = AUTthesis.tex
% در این فایل، عنوان پایان‌نامه، مشخصات خود، متن تقدیمی‌، ستایش، سپاس‌گزاری و چکیده پایان‌نامه را به فارسی، وارد کنید.
% توجه داشته باشید که جدول حاوی مشخصات پروژه/پایان‌نامه/رساله و همچنین، مشخصات داخل آن، به طور خودکار، درج می‌شود.
%%%%%%%%%%%%%%%%%%%%%%%%%%%%%%%%%%%%
% دانشکده، آموزشکده و یا پژوهشکده  خود را وارد کنید
\faculty{دانشکده مهندسی کامپیوتر}
% گرایش و گروه آموزشی خود را وارد کنید
\department{}
% عنوان پایان‌نامه را وارد کنید
\fatitle{توليد موسيقی با محاسبات کوانتومی}
% نام استاد(ان) راهنما را وارد کنید
\firstsupervisor{دکتر مرتضی صاحب‌الزمانی}
%\secondsupervisor{استاد راهنمای دوم}
% نام استاد(دان) مشاور را وارد کنید. چنانچه استاد مشاور ندارید، دستور پایین را غیرفعال کنید.
%\firstadvisor{دکتر بهادر بخشی}
%\secondadvisor{استاد مشاور دوم}
% نام نویسنده را وارد کنید
\name{عرفان }
% نام خانوادگی نویسنده را وارد کنید
\surname{عابدی}
%%%%%%%%%%%%%%%%%%%%%%%%%%%%%%%%%%
\thesisdate{مهر ۱۴۰۰}

% چکیده پایان‌نامه را وارد کنید
\fa-abstract{
امروزه، با نزدیک‌شدن تکنولوژی‌های ساخت سخت‌افزار به محدودیت‌های فیزیکی قانون مور
% \fnote{Moore's law}
و پیش‌رفت روزافزون تکنولوژی‌های فوق سرد
% \fnote{Cryogenics}،
توجه بسیاری به دانشمندان به حوزه‌ی محاسبات کوانتومی معطوف شده است. 
کامپیوترهایی که در حال حاضر به صورت روزمره در حال استفاده هستند، در انجام محاسبات خود از قوانین فیزیک کلاسیک پیروی می‌کنند و به کامپیوترهای کلاسیک معروف هستند.
محاسباتی که توسط کامپیوترهای کوانتومی انجام می‌شوند، بر خلاف کامپیوترهای کلاسیک، تابع قوانین فیزیک کوانتومی هستند؛ به همین علت، بسیاری بر این باور هستند که این گونه کامپیوترها در آینده‌ای نزدیک، قادر به انجام محاسباتی خواهند بود که به سادگی توسط کامپیوترهای کلاسیک ممکن نیست.
یادگیری ماشین کوانتومی
% \fnote{Quantum Machine Learning}
به گروهی از الگوریتم‌های کوانتومی اطلاق می‌شود که همانند یادگیری ماشین کلاسیک، تعدادی پارامتر قابل تنظیم دارند که بهینه‌سازی این پارامترها برای رسیدن به خروجی مطلوب، توسط یک کامپیوتر کلاسیک انجام می‌گیرد.
در عین حال، از همان روزهای اولیه‌ی پیدایش کامپیوترها، بسیاری به دنبال تولید ملودی‌های موسیقی با استفاده از قدرت پردازشی کامپیوترها بوده‌اند و تاکنون، الگوریتم‌های متعددی برای حصول این امر پیشنهاد شده‌اند.
در این پروژه، امکان تولید موسیقی با استفاده از الگوریتم‌های یادگیری ماشین کوانتومی بررسی شده و برنامه‌ی نرم‌افزاری‌ای برای انجام این کار توسعه داده شده است.
% در این پروژه، سعی شده با به کار گیری از یادگیری ماشین کوانتومی، ملودی‌های موسیقی بدیعی تولید شود.
}


% کلمات کلیدی پایان‌نامه را وارد کنید
\keywords{
محاسبات کوانتومی، یادگیری ماشین، تولید موسیقی کامپیوتری
}



\AUTtitle
%%%%%%%%%%%%%%%%%%%%%%%%%%%%%%%%%%
\vspace*{7cm}
\thispagestyle{empty}
\begin{center}
\includegraphics[height=5cm,width=12cm]{figures/besm.jpg}
\end{center}