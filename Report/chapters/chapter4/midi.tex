\section{ماژول 
\lr{Midi}
} \label{sec:midi_module}
ماژول
\lr{Midi}
مسئولیت پیش‌پردازش داده‌ها برای استفاده از مدل‌های یادگیری ماشین کوانتومی را بر عهده دارد.
مجموعه داده‌ی این پروژه، شامل ۹۲ قطعه‌ی موسیقی پیانو به صورت فایل‌های 
\lr{midi}
است، هرکدام از این فایل‌ها، مجموعه‌ای از نت‌ها، آکوردها و زمان پخش آن نت/آکورد از ابتدای قطعه به میلی‌ثانیه است.
این ماژول، با استفاده از کتاب‌خانه‌ی 
\lr{Music21}
تنها یک‌بار پوشه‌ی شامل مجموعه داده‌ها را به صورت کامل بررسی کرده، نت‌های تمامی قطعات را به صورت متوالی در یک لیست ذخیره کرده و در نهایت در یک فایل واحد با نام
\lr{notes.pk}
ذخیره می‌کند. شایان ذکر است که در صورت برخورد با یک آکورد، نت‌های آن را استخراج کرده و همانند چند نت عادی که پشت سر هم پخش می‌شوند، آن‌ها را به لیست نت‌ها اضافه می‌کند.
در عین حال، برای قابل‌فهم کردن نت‌ها برای یک مدل یادگیری ماشین، یک نگاشت ۱-به-۱
از نت‌ها به اعداد طبیعی ساخته می‌شود.
این ماژول سپس در هر بار اجرای کد، با گرفتن پارامتری به نام
\lr{SequenceLength}
تعداد زیادی جفت ورودی و خروجی برای مدل یادگیری ماشین کوانتومی فراهم می‌کند؛ به این صورت:

\begin{equation}
    \begin{gathered}
    input_i = [inotes_{(i)}, ..., inotes_{(SequenceLength + i)}] \\[3pt]
    norm_i = \sqrt{\sum^{SequenceLength+i}_{k=i} (inotes_{(k)})^2 } \\[3pt]
    output_i = [inotes_{(SequenceLength + i + 1)}] \\[3pt]
    where \hso 0 \leq i \leq n - SequenceLength - 1 \hso ; \hso n = \#notes
    \end{gathered}
\end{equation}
\myequations{نحوه تولید ورودی و خروجی ماژول \lr{Midi}}
که در معادله‌ی بالا،
$inotes_{(i)}$
برابر با عدد طبیعی‌ای‌ست که معادل
$i$-
امین نت در مجموعه نت‌هاست.
