\section{ماژول 
\lr{Midi}
} \label{sec:midi_module}

\begin{algorithm}[t]
\caption{استخراج نت‌ها، آکوردها و فرکانس‌های آن‌ها از مجموعه داده} \label{alg:midi_extract}
\lr{
    \begin{algorithmic}
        % \STATE mapping $\leftarrow$ empty dictionary
        % \STATE mapping $\leftarrow$ empty dictionary
        \STATE notes $\leftarrow$ empty list
        \STATE frequencies $\leftarrow$ empty list
        \STATE notes\_file $\leftarrow$ list of notes and chords read from file
        \FOR {new\_part $\in$ notes\_file}
            \IF {new\_part is a Note}
                \STATE notes.add(new\_part.note\_string)
                \STATE frequency = get\_frequency\_from\_note(new\_part)
                \STATE frequencies.add(frequency)
            \ELSIF {new\_part is a Chord}
                \STATE sum\_frequencies = 0
                \FOR {note $\in$ new\_part.notes}
                    \STATE frequency = get\_frequency\_from\_note(note)
                    \STATE sum\_frequencies = sum\_frequencies + frequency
                \ENDFOR
                \STATE sum\_frequencies = sum\_frequencies / new\_part.note\_count
                \STATE notes.add(new\_part.chord\_string)
                \STATE frequencies.add(sum\_frequencies)
            \ENDIF
        \ENDFOR
        \STATE save notes to file: "notes.pk"
        \STATE save frequencies to file: "frequencies.pk"
    \end{algorithmic} 
}
\end{algorithm}
\myalgorithms{
    استخراج داده‌ها از مجموعه داده
}

\begin{algorithm}[t]
\caption{ساخت نگاشت یک به یک از نت‌ها به اعداد طبیعی} \label{alg:midi_mapping}
\lr{
    \begin{algorithmic}
        \STATE notes\_to\_frequencies = map(keys=notes, values=frequencies)
        \STATE mapping $\leftarrow$ empty dictionary
        \STATE notes\_to\_frequencies = sort\_by\_value(notes\_to\_frequencies)
        \STATE sorted\_notes = notes\_to\_frequencies.keys
        \FOR {i := 0 to notes.length}
            \STATE mapping.update(key=i, value=sorted\_notes[i])
        \ENDFOR
    \end{algorithmic} 
}
\end{algorithm}
\myalgorithms{
    ساخت نگاشت یک به یک از نت‌ها به اعداد طبیعی
}


ماژول
\lr{Midi}
مسئولیت پیش‌پردازش داده‌ها برای استفاده از مدل‌های یادگیری ماشین کوانتومی را بر عهده دارد.
مجموعه داده‌ی این پروژه، شامل نود و دو قطعه‌ی موسیقی پیانو به صورت فایل‌های 
\lr{midi}
است، هرکدام از این فایل‌ها، مجموعه‌ای از نت‌ها، آکوردها و زمان پخش آن نت/آکورد از ابتدای قطعه به میلی‌ثانیه است.

\subsection{استخراج داده‌ها از مجموعه داده}
این ماژول، در ابتدا تنها یک‌بار پوشه‌ی شامل مجموعه داده‌ها را بررسی می‌کند و فایل‌های
\lr{midi}
موجود در در پوشه را به ترتیب خوانده و پردازش می‌کند. نتیجه‌ی این پردازش، دو لیست به نام
\lr{notes}
و
\lr{frequencies}
است که تناظر یک به یکی بین این دو لیست وجود دارد؛ به این معنا که اولین فرکانس موجود در لیست فرکانس‌ها، فرکانس صدای تولید شده توسط اولین نت موجود در لیست نت‌ها است.
این ماژول پس از خواندن هر فایل
\lr{midi}
از روی دیسک کامپیوتر، موجودیت‌های حاضر در آن فایل را با استفاده از کتابخانه‌ی
\lr{Music21}
بررسی می‌کند. در صورتی که موجودیت بررسی شده چیزی غیر از نت یا آکورد باشد، به آن توجهی نمی‌کند و در صورتی که آن موجودیت یک نت یا آکورد باشد، آن را به صورت زیر پردازش می‌کند: \\
% \hsm
کتابخانه‌ی
\lr{Music21}
هر نت موجود در یک فایل
\lr{midi}
را به صورت یک نمونه\fnote{Instance} از کلاس
\lr{music21.note.Note}
در نظر می‌گیرد که این کلاس شامل خواص مختلفی از جمله نام آن نت، اکتاو آن نت، فرکانس صدای تولید شده توسط آن نت و مدت زمان پخش آن نت است.
این ماژول در هنگام پردازش یک نمونه از کلاس \\
\lr{music21.note.Note}
نام آن نت را به صورت یک رشته\fnote{String} کاراکتر و فرکانس آن نت را به صورت یک عدد حقیقی پردازش می‌کند که این رشته کاراکتر به لیست
\lr{notes}
و این فرکانس به لیست
\lr{frequencies}
اضافه می‌شوند.
هم‌چنین، این کتابخانه، هر آکورد موجود در یک فایل
\lr{midi}
را به صورت یک نمونه از کلاس
\lr{music21.chord.Chord}
در نظر می‌گیرد که این کلاس شامل خواص مختلفی از جمله نت‌های موجود در آن آکورد و حجم صدای تولید شده در هنگام پخش آن آکورد است.
نت‌های موجود در یک کلاس \\
\lr{music21.chord.Chord}
به صورت لیستی از کلاس‌های
\lr{music21.note.Note}
هستند که همین امر، کار پردازش آکوردها را بسیار آسان‌تر می‌کند.
ماژول
\lr{Midi}
در هنگام پردازش یک نمونه از کلاس \\
\lr{music21.chord.Chord}،
این آکورد را به عنوان یک نت جدید محسوب می‌کند و برای آن رشته کاراکتر و فرکانس خاصی در نظر می‌گیرد.
رشته کاراکتر متناظر یک آکورد به این صورت ساخته می‌شود: نت‌های آن آکورد به صورتی پشت سر هم قرار می‌گیرند که کاراکتر نقطه
('.')
بین آن‌ها قرار گرفته باشد. فرکانس این آکورد نیز به صورت میانگین فرکانس نت‌های موجود در آن آکورد در نظر گرفته می‌شود که همانند مرحله‌ی پردازش نت‌ها، رشته کاراکتر تولید شده به لیست
\lr{notes}
و فرکانس تولید شده به لیست
\lr{frequencies}
اضافه می‌شوند.
فرم شبه‌کدی الگوریتم استخراج نت‌ها و آکوردها از یک فایل به صورت کامل در الگوریتم
\ref{alg:midi_extract}
آمده است.

\subsection{نگاشت داده‌ها به اعداد طبیعی}
ماژول 
\lr{Midi}
به جهت قابل‌فهم کردن نت‌ها برای مدل‌های یادگیری ماشین، یک نگاشت یک به یک از نت‌هایی که در لیست
\lr{notes}
قرار گرفته‌اند به اعداد طبیعی می‌سازد. این نگاشت به صورت زیر تشکیل می‌شود: \\
ابتدا با توجه به تناظری که گفته شد بین لیست
\lr{notes}
و لیست
\lr{frequencies}
وجود دارد و با استفاده از یک جدول درهم‌سازی\fnote{Hash map}، یک نگاشت از نت‌ها به فرکانس‌ها با نام
\lr{notes\_to\_frequencies}
ساخته می‌شود. در مرحله‌ی بعد، به دلیل این‌که جداول درهم‌سازی در زبان برنامه‌نویسی پایتون دارای ترتیب نیز هستند، می‌توان این نگاشت را با توجه به فرکانس‌ها به صورت صعودی مرتب کرد؛ به این معنا که جفت (نت، فرکانس)ی که مقدار فرکانس کم‌تری داشته‌باشد، در ابتدای این نگاشت قرار می‌گیرد. سپس کلیدهای نگاشت
\lr{notes\_to\_frequencies}
به صورت جداگانه استخراج می‌شوند و به عدد طبیعی متناظر مکان‌شان در ترتیب موجود نگاشت می‌شوند که این نگاشت، در متغیر
\lr{mapping}
ذخیره می‌شود.
فرم شبه‌کدی الگوریتم ساخت نگاشت یک به یک از نت‌ها به اعداد طبیعی در الگوریتم
\ref{alg:midi_mapping}
آمده است.



این ماژول سپس در هر بار اجرای کد، با گرفتن پارامتری به نام
\lr{SequenceLength}
تعداد زیادی جفت ورودی و خروجی برای مدل یادگیری ماشین کوانتومی فراهم می‌کند؛ به این صورت:

\begin{equation}
    \begin{gathered}
    input_i = [inotes_{(i)}, ..., inotes_{(SequenceLength + i)}] \\[3pt]
    norm_i = \sqrt{\sum^{SequenceLength+i}_{k=i} (inotes_{(k)})^2 } \\[3pt]
    output_i = [inotes_{(SequenceLength + i + 1)}] \\[3pt]
    where \hso 0 \leq i \leq n - SequenceLength - 1 \hso ; \hso n = \#notes
    \end{gathered}
\end{equation}
\myequations{نحوه تولید ورودی و خروجی ماژول \lr{Midi}}
که در معادله‌ی بالا،
$inotes_{(i)}$
برابر با عدد طبیعی‌ای‌ست که معادل
$i$-
امین نت در مجموعه نت‌هاست.
