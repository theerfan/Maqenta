\section{
ماژول
\lr{QuGAN}
}
ماژول
\lr{QuGAN}
بر اساس شبکه‌های زایای دشمن‌گونه‌ی کوانتومی که در بخش
\ref{sec:qugan}
معرفی شد طراحی شده است.
این ماژول نیز همانند ماژول
\lr{QLSTM}
از ماژول
\lr{Midi}
برای پیش‌پردازش داده‌ها استفاده می‌کند.

این ماژول به طور کلی شامل سه زیرمدار کوانتومی است. از ترکیب‌های مختلف این سه زیرمدار، سه مدار کوانتومی تشکیل می‌شود که توضیح کارکرد آن‌ها در زیر آمده است:

\begin{itemize}
    \item زیرمدار اول که در کد پروژه توسط تابع
    \lr{encode\_music}
    ساخته می‌شود، با دریافت نت‌هایی به عنوان ورودی، آن‌ها را با استفاده از تابع
    \lr{AmplitudeEmbedding}
    در کیوبیت‌های سیستم کدگذاری می‌کند.
    
    \item زیرمدار دوم که زیرمداری پارامتریک است و در کد پروژه توسط تابع
    \lr{discriminator}
    ساخته می‌شود، با استفاده از تابع
    \lr{RandomLayers}
    لایه‌هایی پارامتریک تولید می‌کند. این لایه‌ها سعی در تشخیص واقعی یا ساختگی بودن داده‌های موجود در کیوبیت‌ها دارند و عمل‌کرد متمایزکنندگی شبکه را پیاده‌سازی می‌کنند.
    خروجی این زیرمدار، آرایه‌ای متشکل از امیدریاضی اندازه‌گیری تک‌تک کیوبیت‌های سیستم به صورت مجزا است.
    
    \item زیرمدار سوم، همانند زیر مدار دوم مداری پارامتریک است، اما سعی در تولید داده‌هایی ساختگی از روی کیوبیت‌هایی که مقداری نویز به عنوان داده‌ی اولیه بر روی آن‌ها کدگذاری شده‌اند دارد. این زیرمدار در کد پروژه توسط تابع
    \lr{music\_generator}
    ساخته می‌شود و عملکرد زایایی شبکه را پیاده‌سازی می‌کند.
\end{itemize}

و مدارهای کوانتومی این ماژول، به این شکل ساخته شده‌اند:
\begin{itemize}
    \item
    مدار اول که
    \lr{real\_music\_discriminator}
    نام دارد،
    ابتدا به کمک تابع
    \lr{encode\_music}
    تعدادی نت از مجموعه داده گرفته و آن‌ها را در کیوبیت‌های سیستم کدگذاری می‌کند، سپس زیرمدار
    \lr{discriminator}
    سعی می‌کند با بهینه‌سازی پارامترهای خود تشخیص دهد آیا داده‌ها واقعی هستند یا خیر.
    نمایش دیداری این مدار در شکل 
    \ref{fig:real_disc}
    آمده است.
    
    \item
    مدار دوم که
    \lr{generated\_music\_discriminator}،
    نام دارد، ابتدا با گرفتن مقداری نویز به عنوان ورودی، آن نویزها را توسط
    \lr{encode\_music}
    در سیستم کدگذاری می‌کند، سپس زیرمدار \\
    \lr{music\_generator}
    با بهینه‌سازی پارامترهای خود، اقدام به ساخت داده‌ی جدیدی از روی نویز کدگذاری شده می‌کند و در نهایت، زیرمدار
    \lr{discriminator}
    سعی می‌کند با بهینه‌سازی پارامترهای خود، تشخیص دهد آیا داده یکی از داده‌های واقعی است یا خیر.
    نمایش دیداری این مدار در شکل 
    \ref{fig:gen_disc}
    آمده است.
    
    \item
    مدار سوم که
    \lr{final\_music\_generator}
    نام دارد، با ترکیب زیرمدار‌های
    \lr{encode\_music}
    و \\
    \lr{music\_generator}
    ابتدا مقداری نویز را در سیستم کدگذاری کرده و سپس آرایه‌ای متشکل از امیدریاضی اندازه‌گیری تک‌تک کیوبیت‌های سیستم را به عنوان خروجی تولید می‌کند که نتیجه‌ی نهایی الگوریتم، از این خروجی ساخته می‌شود.
    نمایش دیداری این مدار در شکل 
    \ref{fig:final_gen}
    آمده است.
    
\end{itemize}

در شکل‌های این مدارها، زیرمدار
\lr{encode\_music}
با نام
$EM(x)$،
زیرمدار
\lr{discriminator}
با نام
$Disc(\theta)$
و زیرمدار
\lr{generator}
با نام
$Gen(\phi)$
مشخص شده است.



\begin{figure}
    \centering
    \begin{quantikz}
            \lstick[wires=3]{$\ket{0}^{\otimes n\_qubits}$} & \gate[wires=3][2cm]{EM(x)} & \gate[wires=3][2cm]{Disc(\theta)} & \qw & \meter{}  \\
            & \qw & \qw & \qw & \meter{} \\
            & \qw & \qw & \qw & \meter{}
    \end{quantikz}
    \caption{
    نمایش دیداری مدار
    \lr{real\_music\_discriminator}
    }
    \label{fig:real_disc}
    \bigskip
    \begin{quantikz}
            \lstick[wires=3]{$\ket{0}^{\otimes n\_qubits}$} & \gate[wires=3][2cm]{EM(x)} & \gate[wires=3][2cm]{Gen(\phi)} &\gate[wires=3][2cm]{Disc(\theta)} & \qw & \meter{}  \\
            & \qw & \qw & \qw & \qw & \meter{} \\
            & \qw & \qw & \qw & \qw & \meter{}
    \end{quantikz}
    \caption{
    نمایش دیداری مدار
    \lr{generated\_music\_discriminator}
    }
    \label{fig:gen_disc}
    \bigskip
    \begin{quantikz}
            \lstick[wires=3]{$\ket{0}^{\otimes n\_qubits}$} & \gate[wires=3][2cm]{EM(x)} & \gate[wires=3][2cm]{Gen(\phi)} & \qw & \meter{}  \\
            & \qw & \qw & \qw & \meter{} \\
            & \qw & \qw & \qw & \meter{}
    \end{quantikz}
    \caption{
    نمایش دیداری مدار کوانتومی
    \lr{final\_music\_generator}
    }
    \label{fig:final_gen}
\end{figure}

نکته‌ی اصلی این است که بعد از هربار اجرای مدارهای اول و دوم، وزن‌های زیرمدار
\lr{discriminator}
آن‌ها با هم همگام می‌شوند، چراکه در کل برنامه باید تنها یک
\lr{discriminator}
وجود داشته‌باشد.
الگوریتم کلی اجرا شده در هنگام آموزش این ماژول، دارای چند حلقه است. در هر مرحله از حلقه‌ی اول که
\lr{discriminator\_iteration}
نام دارد، ابتدا با استفاده از الگوریتم گرادیان کاهشی، یک گام در بهینه‌سازی پارامترهای زیرمدار
\lr{discriminator}
 برداشته می‌شود، سپس پارامترهای این زیرمدار در دو مدار
\lr{real\_music\_discriminator}
و \\
\lr{generated\_music\_discriminator}
همگام‌سازی می‌شوند.
در حلقه‌ی دوم که
\lr{generator\_iteration}
نام دارد، بهینه‌سازی پارامترهای زیرمدار 
\lr{music\_generator}
انجام می‌شود و در هر مرحله از اجرای حلقه‌ی سوم، ابتدا حلقه‌ی
\lr{discriminator\_iteration}
و سپس حلقه‌ی
\lr{generator\_iteration}
اجرا می‌شوند.
فرم شبه‌کدی این الگوریتم در الگوریتم
\ref{alg:qugan_iteration}
آمده است که در این الگوریتم، تابع
\lr{sync\_disc}
تابعی‌ست که پارامترهای موجود در زیرمدارهای
\lr{discriminator}
موجود در مدارهای
\lr{real\_music\_discriminator}
و
\lr{generated\_music\_discriminator}
را همگام می‌سازد.
هم‌چنین شایان ذکر است که بهینه‌سازی پارامترها در این الگوریتم، با استفاده از توابع هزینه‌ی معرفی شده در معادله‌ی
\ref{eqn:qugan_cost}
انجام می‌گیرد.

\begin{algorithm}[t]
\caption{نحوه کارکرد کلی ماژول \lr{QuGAN}} \label{alg:qugan_iteration}
\lr{
    \begin{algorithmic}
        \STATE real\_music\_discriminator \hso $\leftarrow$ pre-generated quantum circuit
        \STATE generated\_music\_discriminator \hso $\leftarrow$ pre-generated quantum circuit
        \STATE steps \hso $\leftarrow$ number of iterations for the outer loop
        \STATE n\_iterations \hso $\leftarrow$ number of iterations for each inner loop
        \STATE discriminator = real\_music\_discriminator.discriminator
        \STATE generator = generated\_music\_discriminator.generator
        \FOR {$i=0 \to steps$}
            \FOR {$j=0 \to n\_iterations$}
                \STATE real\_music\_discriminator.optimize$(discriminator)$
                \STATE sync\_disc$(real\_music\_discriminator, generated\_music\_discriminator)$
            \ENDFOR
            \FOR {$j=0 \to n\_iterations$}
                \STATE generated\_music\_discriminator.optimize$(generator)$
            \ENDFOR
        \ENDFOR
    \end{algorithmic} 
}
\end{algorithm}
\myalgorithms{
    نحوه کارکرد کلی ماژول \lr{QuGAN}
}

لازم به ذکر است که در این ماژول، نویزی که به مدارهای
\lr{generated\_music\_discriminator}
و
\lr{final\_music\_generator}
داده می‌شود، تعدادی نت تصادفی از مجموعه داده است.
پس از بهینه‌سازی پارامترهای زیرمدارهای
\lr{discriminator}
و
\lr{music\_generator}،
پارامترهای زیرمدار
\lr{music\_generator} 
موجود در مدار 
\lr{final\_music\_generator}
با مدار
\lr{generated\_music\_discriminator}
همگام‌سازی شده و پس از پس‌پردازش خروجی‌هایی که از مدار
\lr{final\_music\_generator}
تولید می‌شود، به مجموعه‌ای از نت‌های موسیقی دست می‌یابیم.

\subsection{پس‌پردازش}

به همان دلیل ارائه شده در بخش
\ref{sec:qlstm_post}،
داده‌های این ماژول نیز نیاز به پس‌پردازش دارد، اما این ماژول برای تولید قطعات موسیقی آهنگین‌تر، نیاز به پس‌پردازش متفاوتی دارد.
پس‌پردازش استفاده شده در این ماژول طبق الگوریتم
\ref{alg:quganpost}
عمل می‌کند.
در این الگوریتم، همانند پس‌پردازش ماژول
\lr{QLSTM}،
متغیرهای
$input_i$
و
$norm_i$
طبق معادله‌ی
\ref{eqn:midi_data}
تعریف شده‌اند،
$raw\_model\_output_i$
خروجی اولیه‌ای است که از مدار کوانتومی گرفته می‌شود و
$model\_output_i$
خروجی نهایی بعد از پس‌پردازش است.
منظور از 
\lr{mapping}
نیز همان نگاشت یک به یک از نت‌ها به اعداد طبیعی است.

در نهایت، تابع
\lr{generate\_notes}
با گرفتن پارامتر
\lr{n\_notes}
با چندین‌بار اجرای الگوریتم‌های یادگیری ماشین و پس‌پردازش، به تعداد
\lr{n\_notes}
نت موسیقی جدید تولید کرده و با قرار دادن فاصله‌هایی به اندازه‌ی پنجاه میلی‌ثانیه بین آن‌ها، نتایج حاصل را در یک فایل با پسوند
\lr{midi}
ذخیره می‌کند.


\begin{algorithm}[t]
\caption{پس‌پردازش ماژول \lr{QuGAN}} \label{alg:quganpost}
\lr{
    \begin{algorithmic}
        \STATE $raw\_model\_output_i = model(input_i)$
        \STATE $model\_output_i = (raw\_model\_output_i + 1) * norm_i$
        \STATE $counter = 1$
        \WHILE {$model\_output_i \not\in mapping.keys()$}
        \STATE $model\_output_i = model\_output_i * counter / (counter+1)$
        \STATE $model\_output_i = model\_output_i.to\_int()$
        \STATE $counter = counter + 1$
        \ENDWHILE
    \end{algorithmic} 
}
\end{algorithm}
\myalgorithms{
    پس‌پردازش ماژول
    \lr{QuGAN}
}