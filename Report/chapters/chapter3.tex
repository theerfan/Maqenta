\chapter{پیشینه موضوع}
%\thispagestyle{empty}

\section{سرویس‌های مبتنی بر موقعیت مکانی}

نرم‌افزارها و سامانه‌هایی که سرویس‌هایی مبتنی بر موقعیت مکانی ارائه می‌دهد، ال‌بی‌اس\footnote{\lr{LBS}} نامیده می‌شوند. بطورکلی ال‌بی‌اس به تمام خدمات نرم‌افزاری گفته می‌شود که از اطلاعات و داده‌های جغرافیایی استفاده می‌کنند تا خدمت یا اطلاعاتی را به کاربران ارائه دهند\cite{hist:1}.

این نرم‌افزارها لزوما برای پیدا کردن فروشگاه‌های اطراف نیستند و انواع مختلفی دارند. برای مثال می‌توانند در حوزه‌های سلامت، جستجوی اشیا (حتی درون محیط‌های سربسته)، سرگرمی، کار، زندگی شخصی و... مورد استفاده قرار بگیرند. از مثال‌های پرکاربرد حوزه‌هایی که در آن‌ها ال‌بی‌اس‌ها استفاده می‌شوند می‌توان به شبکه‌های اجتماعی، بازاریابی و تبلیغات مبتنی بر مکان و سیستم‌های ره‌گیری و نظارت اشاره کرد. همچنین اگر در راستای بازاریابی از آن استفاده شود، می‌تواند در حوزه تجارت همراه\footnote{\lr{Mobile commerce}} بسیار مورد استفاده قرار بگیرد\cite{hist:1}.

در ادامه به ذکر چند نمونه از حوزه‌هایی که در آن‌ها ال‌بی‌اس‌ها قابل استفاده هستند می‌پردازیم.

\section{شبکه‌های اجتماعی مبتنی بر مکان}


شبکه‌های اجتماعی مبتنی بر مکان یا ال‌بی‌اس‌ان‌\footnote{\lr{LBSN}}ها شبکه‌های اجتماعی هستند که بر مبنای موقعیت مکانی کاربران کار می‌کنند؛ در اینگونه شبکه‌ها، چندان اهمیتی ندارد که هر فرد به تنهایی چه نقشی دارد، بلکه تمامی افراد می‌توانند در آن دخیل باشند؛ از فروشنده‌های لباس گرفته تا پمپ‌های بنزین و حتی اشخاص حقیقی. برای مثال می‌توان به پروژه \lr{Foursquare}\cite{hist:2} (که بعدها تحت عنوان \lr{Swarm}\cite{hist:3} مجددا پیاده‌سازی شد) اشاره کرد.

\begin{figure}[b]
	\centering
	\includegraphics[scale=.1]{foursquare}
	\includegraphics[scale=.2]{swarm}
	\caption{لوگوی فوراسکوئر و سوآرم}
	\label{fig:foursquare}
\end{figure}

\newpage

\section{آب و هوا مبتنی بر مکان}

برخی نرم‌افزارها از موقعیت دقیق کاربر برای نمایش آب و هوا با دقت بسیار بالاتر از دقت‌های مرسوم استفاده می‌کنند؛ همچنین آن‌ها می‌توانند دارای امکاناتی باشند تا تغییرات بسیار نزدیک (در حد کمتر از 1 دقیقه) آب و هوا را بطور لحظه‌ای با استفاده از یک اعلان ، به کاربر هشدار دهد. از اینگونه سامانه‌ها می‌توان به دارک اسکای\footnote{\lr{DarkSky}}\cite{hist:4} اشاره کرد.

\begin{figure}[H]
	\centering
	\includegraphics[scale=1]{darksky}
	\caption{لوگوی دارک اسکای}
	\label{fig:darksky}
\end{figure}

\section{سرگرمی مبتنی بر مکان}

احتمالا نام بازی بسیار معروف پوکیمون‌گو\footnote{\lr{Pokemon Go}}\cite{hist:5} را شنیده‌اید! این بازی در این دسته قرار می‌گیرد؛ در این بازی، کاربران با استفاده از فناوری واقعیت افزوده\footnote{\lr{AR (Augmented reality)}} و استفاده از موقعیت مکانی واقعی، در دنیای حقیقی بدنبال اشیاء مجازی می‌گردند!

\begin{figure}[b]
	\centering
	\includegraphics[scale=.3]{pokemon}
	\caption{لوگوی پوکیمون}
	\label{fig:pokemon}
\end{figure}

\section{سفارش‌دهی مبتنی بر مکان}

گروهی از این سامانه‌ها، بستری برای در صف قرار گرفتن مشتریان یک رستوران یا فروشگاه ایجاد می‌کنند؛ به این‌صورت که در ابتدا مشتری بدون اینکه در یک فروشگاه یا رستوران حضور داشته باشد، سفارش خود را ثبت می‌کند؛ سپس هر وقت سفارش حاضر گردد، اعلانی برای او نمایش داده می‌شود تا نسبت به حضور در محل فروشگاه اقدام کند. وقتی به محل فروشگاه نزدیک شود، بطور خودکار به فروشنده اعلام می‌شود که مشتری در حال رسیدن است و فروشنده سفارش را برای تحویل نهایی، آماده می‌کند. از این گروه می‌توان به نرم‌افزار \lr{Curbside}\cite{hist:6} اشاره کرد.

\begin{figure}[H]
	\centering
	\includegraphics[scale=1]{curbside}
	\caption{لوگوی کربساید}
	\label{fig:curbside}
\end{figure}


\section{کار پیشِ‌رو}

سامانه‌ای که در این پروژه قصد ایجاد آن را داریم، یک سکو\footnote{\lr{Platform}}ی کاربردی خواهد بود که کار آن، ایجاد ارتباط به‌موقع بین فروشندگان محصولات مختلف و علاقه‌مندان خرید آن محصولات است. در واقع می‌توان گفت این سامانه ترکیبی از یک شبکه اجتماعی مبتنی بر مکان (بدلیل سکو بودن آن) و سامانه‌های سفارش‌دهی مبتنی بر مکان است.
