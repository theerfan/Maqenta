
\chapter{مقدمه}

% \section{محاسبات کوانتومی}

با گذشت چهل سال از طرح ایده‌ی محاسبات کوانتومی توسط ریچارد فاینمن
\cite{feynman}،
توجه جهان به کامپیوترهای کوانتومی در بالاترین سطح خود قرار دارد
عمده‌ی این توجه پس از ابداع الگوریتم فاکتورگیری شور
\fnote{Shor's factorization algorithm}
و الگوریتم جست‌وجوی گروور 
\fnote{Grover's quantum search algorithm}
در دهه‌ی نود میلادی آغاز شده است. از آن روز تاکنون، تلاش‌های بسیار زیادی برای طراحی الگوریتم‌های جدید برای کاربردهای متنوع و دسترسی به برتری محاسباتی کوانتومی در حال انجام هستند. در زمان تحریر این گزارش، شرکت‌های بسیاری از جمله آی‌بی‌ام، مایکروسافت و گوگل در حال سرمایه‌گذاری‌های جدی بر روی ساخت کامپیوترهای کوانتومی هستند. به عنوان مثال، در سال ۲۰۱۹،
شرکت گوگل آزمایشی 
\cite{google_supremacy}
انجام داد که نشان‌گر برتری محاسباتی کوانتومی بود، به این معنا که پیچیدگی زمانی انجام این آزمایش در حالت کلاسیک بسیار بزرگ‌تر از حالت کوانتومی است.

با این وجود، به علت محدودیت تعداد کیوبیت\footnote{معادل کوانتومی یک بیت کلاسیک}
های کامپیوترهای کوانتومی فعلی، تلاش‌های اندکی در راستای تولید ملودی‌های موسیقی با استفاده از الگوریتم‌های کوانتومی صورت گرفته‌است. دو الگوریتمی که تاکنون در این راستا تحت بررسی قرار گرفته‌اند، تولید موسیقی با استفاده از تولیدکننده‌ی اعداد تصادفی
و پیمایش گراف 
\cite{miranda}
بوده اند، اما به نظر می‌رسد به علت وجود معادلات موجی در فیزیک کوانتومی و این حقیقت که موسیقی در اصل تشکیل شده از موج‌های صوتی است، در آینده می‌توان ارتباطات خیلی بیش‌تری بین تولید موسیقی و محاسبات کوانتومی کشف کرد.

\newpage
\section{هدف پروژه}
% موضوع این پایان‌نامه، در گام نخست بررسی شیوه‌های موجود استفاده از الگوریتم‌های کوانتومی برای تولید موسیقی؛ و در گام دوم، معرفی دو شیوه جدید برای تولید موسیقی با استفاده از یادگیری ماشین کوانتومی است.
هدف این پروژه، طراحی نرم‌افزاری است که بتواند با استفاده از الگوریتم‌های موجود یادگیری ماشین کوانتومی و یک  مجموعه داده از برخی موسیقی‌های موجود، اقدام به تولید ملودی موسیقی جدیدی کند.
مجموعه داده‌ی این پروژه به صورت سری گسسته‌ای از نت‌ها و آکوردهای موسیقی است که به صورت یک فایل 
\lr{midi}
ذخیره شده‌اند.
عمده‌ی نرم‌افزارهای پخش محتوای چندرسانه‌ای، قابلیت پخش موسیقی با استفاده از فایل‌های 
\lr{midi}
به عنوان ورودی را دارند، اما
 شایان ذکر است که این نوع ذخیره‌سازی، با ذخیره‌سازی‌های شناخته‌شده‌تر محتوای صوتی مانند فایل‌های 
 \lr{mp3}
 متفاوت است؛ چراکه فایل‌های
 \lr{mp3}
 با استفاده از پردازش سیگنال و تبدیل فوریه اقدام به ذخیره‌سازی می‌کنند.
 

\section{اجزاء پروژه}
پیاده‌سازی این پروژه به‌صورت کلی شامل ۳ بخش می‌شود:
\begin{enumerate}
    \item ماژول
    \lr{Midi}
    که مسئولیت پردازش داده‌های موسیقی را بر عهده دارد. این ماژول تعدادی فایل که متشکل از چندین  را به صورتی که به عنوان ورودی برای الگوریتم کوانتومی به راحتی قابل استفاده باشند در می‌آورد. هم‌چنین، این ماژول پس از دریافت خروجی تولید شده به وسیله‌ی اجرای الگوریتم یادگیری ماشین کوانتومی بر روی ورودی‌ها، فایل ملودی جدیدی را تولید می‌کند.
    \item ماژول
    \lr{QLSTM}
    که با استفاده از الگوریتم حافظه‌ی طولانی کوتاه-مدت کوانتومی
    \cite{chen_qlstm}
    ، اقدام به تولید ملودی‌های موسیقی می‌کند.
    
    \item ماژول
    \lr{QuGAN}
    که با استفاده از الگوریتم شبکه‌ی زایای دشمن‌گونه کوانتومی
    \cite{lloyd_qugan} \cite{zoufal_qugan}
    ، اقدام به تولید ملودی‌های موسیقی می‌کند.

\end{enumerate}

\section{ابزارهای مورد استفاده برای پیاده‌سازی}

برای پیاده‌سازی اجزای مختلف این پروژه، از چند کتاب‌خانه‌ی نرم‌افزاری استفاده شده که در فصول بعدی نقش آن‌ها به صورت کامل شرح داده خواهد شد، اما در این بخش صرفا به عنوان مقدمه از آن‌ها نام برده می‌شود:
\lr{
\begin{itemize}
	\item \lr{Music21}
	\item \lr{PyTorch}
	\item \lr{PennyLane}
\end{itemize}
}