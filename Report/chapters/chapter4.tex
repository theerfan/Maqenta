\chapter{پیاده‌سازی و نتایج نو}

در این پروژه، برای (تا جایی که نویسنده اطلاع دارد) اولین بار از الگوریتم‌های یادگیری ماشین کوانتومی برای تولید موسیقی استفاده شده و همان‌طور که در بخش
\ref{sec:parts}
اشاره شد، شامل سه ماژول 
\lr{Midi, QLSTM} 
و
\lr{QuGAN}
است که در این فصل توضیحات کامل عمل‌کرد آن‌ها شرح داده می‌شود.

\section{ماژول 
\lr{Midi}
}
مجموعه داده‌ی این پروژه، شامل ۹۲ قطعه‌ی موسیقی پیانو به صورت فایل‌های 
\lr{midi}
است، هرکدام از این فایل‌ها، مجموعه‌ای از نت‌ها، آکوردها و زمان پخش آن نت/آکورد از ابتدای قطعه به میلی‌ثانیه است.
این ماژول، تنها یک‌بار پوشه‌ی شامل مجموعه داده‌ها را به صورت کامل بررسی کرده و نت‌های تمامی قطعات را به صورت متوالی در یک لیست ذخیره کرده و در نهایت در یک فایل واحد با نام
\lr{notes.pk}
ذخیره می‌کند. شایان ذکر است که در صورت برخورد با یک آکورد، نت‌های آن را استخراج کرده و همانند چند نت عادی که پشت سر هم پخش می‌شوند، آن‌ها را به لیست نت‌ها اضافه می‌کند.
این ماژول سپس در هر بار اجرای کد، با گرفتن پارامتری به نام
\lr{SequenceLength}
تعداد زیادی جفت ورودی و خروجی برای مدل یادگیری ماشین کوانتومی فراهم می‌کند؛ به این صورت:

\begin{equation}
    \begin{gathered}
    input_i = [i, ..., SequenceLength + i] \\[3pt]
    output_i = [SequenceLength + i + 1] \\[3pt]
    where \hso 0 \leq i \leq n - SequenceLength - 1 \hso ; \hso n = \#notes
    \end{gathered}
\end{equation}
\myequations{نحوه تولید ورودی و خروجی ماژول \lr{Midi}}

\section{ماژول
\lr{QLSTM}
} 
این ماژول بر اساس حافظه‌های طولانی کوتاه-مدت کوانتومی که در بخش
\ref{sec:qlstm}
معرفی شدند طراحی شده است.
در کتاب‌خانه‌ی 
\lr{PyTorch}
هر مدل تمرین‌پذیر یادگیری ماشین، زیرکلاسی از کلاس
\lr{torch.nn.Module}
است. به همین دلیل، هر واحد بازگشتی کوانتومی این الگوریتم در کلاس
\lr{QLSTMCell}
تعریف شده که زیرکلاسی از کلاس 
\lr{torch.nn.Module}
است.
ورودی‌های مهمی که برای ساختن نمونه‌ای از این کلاس لازم است به شرح زیر هستند:
\begin{itemize}
    \item 
    \lr{n\_qubits}
    که تعداد کیوبیت‌های استفاده شده در مدارهای کوانتومی این واحد بازگشتی است.
    \item
    \lr{n\_qlayers}
    که تعداد لایه‌های موجود در قسمت پارامتردار مدارهای کوانتومی را تعیین می‌کند.
    \item
    \lr{input\_size}
    که ابعاد داده‌های ورودی به واحد بازگشتی را تعیین می‌کند.
    \item
    \lr{hidden\_size}
    که ابعاد داده‌ی خروجی واحد بازگشتی را تعیین می‌کند.
\end{itemize}