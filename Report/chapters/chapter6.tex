\chapter{خروجی و آزمون‌ها}

در این فصل سعی شده است تا با ارائه تعدادی تصاویر از محیط سامانه و عملکرد آن، خروجی سامانه و همچنین برخی از آزمون‌های تولید شده نیز ارائه شود. البته تعداد بسیار بیشتری تصویر و بخش قابل ارائه بود اما در این‌صورت حجم این گزارش بسیار زیاد می‌شد. همچنین  بدلیل این‌که جزئیات کار سامانه در فصول قبلی شرح داده شده از توضیح تصاویر فوق صرف‌نظر شده است.

\section{سامانه اصلی}

\subsection{ساختار فایل‌ها}

\begin{figure}[H]
	\centering
	\includegraphics[scale=0.5]{files}
	\caption{فایل‌های سامانه اصلی}
	\label{fig:files}
\end{figure}

\subsection{خروجی سامانه اصلی}

\begin{figure}[H]
	\centering
	\includegraphics[scale=0.45]{backend}
	\caption{درخواست ازمایشی دریافت پیشنهاد مبتنی بر مکان}
	\label{fig:backend}
\end{figure}


\newpage

\subsection{خروجی نرم‌افزار کاربردی}

\begin{figure}[H]
	\centering
	\includegraphics[scale=0.1]{1}
	\includegraphics[scale=0.1]{2}
	\includegraphics[scale=0.1]{3}
	\caption{به ترتیب از راست، لیست صفحه اصلی، لیست محصولات یک فروشگاه، صفحه یک محصول}
	\label{fig:app1}
\end{figure}

\begin{figure}[H]
	\centering
	\includegraphics[scale=0.1]{4}
	\includegraphics[scale=0.1]{5}
	\includegraphics[scale=0.1]{6}
	\caption{به ترتیب از راست، صفحه افزودن فروشگاه، صفحه تغییر یک محصول، صفحه مشاهده و آمارگیری لیست کوپن‌ها}
	\label{fig:app2}
\end{figure}

\section{آزمون سامانه}

\begin{figure}[H]
	\centering
	\includegraphics[scale=0.3]{test1}
	\caption{آزمون واحد تابع خطی‌ساز اشیا}
	\label{fig:test1}
\end{figure}


\begin{figure}[H]
	\centering
	\includegraphics[scale=0.4]{test2}
	\caption{آزمون واحد تولید بلیت با \lr{JWT}}
	\label{fig:test2}
\end{figure}


\begin{figure}[H]
	\centering
	\includegraphics[scale=0.3]{test3}
	\caption{آزمون واحد توابع کمکی کار با آرایه \lr{ArrayFilter}}
	\label{fig:test3}
\end{figure}



