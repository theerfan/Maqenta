\chapter{نتیجه‌گیری}

در این پروژه‌ی کارشناسی، سعی شد تا به بررسی محاسبات کوانتومی، یادگیری ماشین، تئوری موسیقی و تلاقی این سه حوزه برای تولید موسیقی‌های بدیع پرداخته شود. به نظر می‌رسد که تا به حال تلاش‌های کمی در راستای بررسی این تلاقی صورت گرفته است. با این وجود که به خاطر خواص موجی فیزیک کوانتومی و خواص موجی نت‌های موسیقی، احتمال داده می‌شود که پتانسیل‌های بسیار زیادی در این زمینه وجود داشته باشد.
ماحصل این پروژه، برنامه‌ی نرم‌افزاری‌ای است که با استفاده از یک مجموعه‌داده و چندین کتاب‌خانه‌ی پرکاربرد، موسیقی‌های بدیعی تولید می‌کند.

یک نکته حائز اهمیت این است که یادگیری این الگوریتم‌ها به نسبت سریع است و لزوما به تعداد کیوبیت‌های زیاد یا مدارهای عمیق برای تولید قطعات موسیقی آهنگین احتیاجی ندارد.
نکته حائز اهمیت دیگر این است که الگوریتم‌های پیاده‌سازی شده در این کتاب‌خانه، محدود به همین مجموعه داده نیستند و می‌توان با تزریق مجموعه داده‌ی دیگری (به عنوان مثال، اصوات متعلق به سازهای دیگر، همچون گیتار و درام) به آن، نتایج متفاوتی دریافت کرد.

در نهایت، کد این پروژه و ۴ نمونه موسیقی تولید شده با استفاده از آن را می‌توان در مخزن گیت‌هاب
\lr{Maqneta}
\cite{Maqenta}
مشاهده کرد.

\section{کارهای آینده}
همان‌طور که اشاره شد، هنوز جنبه‌های بسیاری از این حوزه کشف‌نشده باقی مانده‌اند. به عنوان مثال، الگوریتم متفاوتی
\cite{bausch2020recurrent}
برای شبکه‌های عصبی بازگشتی کوانتومی نیز ارائه شده‌است که در این پروژه، امکان تولید موسیقی با استفاده از این الگوریتم بررسی نشده است. هم‌چنین، الگوریتم‌های کلاسیکی برای تولید موسیقی با استفاده از یادگیری تقویتی\fnote{Reinforcement learning} و یادگیری انتقالی\fnote{Transfer learning} نیز وجود دارند و با توجه به این‌که نسخه‌ای کوانتومی از هر دوی این الگوریتم‌ها موجود است
\cite{Mari_qtransfer}
\cite{Daoyi_QReinforce}
احتمالا بتوان با بررسی آن‌ها نیز به تولید موسیقی پرداخت.

در حال حاضر، فعالیت‌هایی در این حوزه در جریان است. به عنوان مثال، کنفرانس
\lr{ISQCMC}
\cite{ISQCMC}
که در اواخر ماه نوامبر سال میلادی جاری برگزار خواهد شد، به برخی جنبه‌های دیگر تولید موسیقی با استفاده از محاسبات کوانتومی، همانند
تولید موسیقی با استفاده از اتوماتاهای سلولی کوانتومی\fnote{Quantum cellular automata} خواهد پرداخت.
