%%%  کلاس AUTthesis، نسخه آبان 1397
%%%   دانشگاه صنعتی امیرکبیر                 http://www.aut.ac.ir
%%%  تالار گفتگوی پارسی‌لاتک،       http://forum.parsilatex.com
%%%   آپدیت شده در آبان 95
%%%   پشتیبانی و راهنمایی          badali_farhad@yahoo.com
%%%
%%%   بازبینی و اصلاح شده در آبان ماه 1397
%%%  Tested via TeXstudio in TeXlive 2014-2018.
%%%

%-----------------------------------------------------------------------------------------------------
%        روش اجرا.: 2 بار F1 ، 2 بار  F11(به منظور تولید مراجع) ، دوبار Ctrl+Alt+I (به منظور تولید نمایه) و دو بار F1 -------> مشاهده Pdf
%%%%%%%%%%%%%%%%%%%%%%%%%%%%%%%%%%%%%%%%%%%%%%%%%%%%%%
%   TeXstudio as your IDE
%%  برای compile در TeXstudio تنها کافی است منوی Options->Configure TeXstudio را زده و در پنجره Configure TeXstudio در بخش Build گزینه Default Compiler را به XeLaTeX تغییر دهید. سند شما به راحتی compile خواهد شد.
%   F1 & F5 : Build & view
%   F6      : Compile
%   F7      : View
%   --------------
%%%%%%%%%%%%%%%%%%%%%%%%%%%%%%%%%%%%%%%%%%%%%%%%%%%%%%
%        اگر قصد نوشتن رساله دکتری را دارید، در خط زیر به جای msc،
%      کلمه phd را قرار دهید. کلیه تنظیمات لازم، به طور خودکار، اعمال می‌شود.
%%% !TEX TS-program = XeLaTeX
\documentclass[oneside,bsc,12pt]{AUTthesis}
%       فایل commands.tex را حتماً به دقت مطالعه کنید؛ چون دستورات مربوط به فراخوانی بسته زی‌پرشین 
%       و دیگر بسته‌ها و ... در این فایل قرار دارد و بهتر است که با نحوه استفاده از آنها آشنا شوید. توجه شود برای نسخه نهایی پایان‌نامه حتماً hyperref را 
%        غیرفعال کنید.


% در این فایل، دستورها و تنظیمات مورد نیاز، آورده شده است.
%-------------------------------------------------------------------------------------------------------------------
% در ورژن جدید زی‌پرشین برای تایپ متن‌های ریاضی، این سه بسته، حتماً باید فراخوانی شود.
\usepackage{amsthm,amssymb,amsmath,amsfonts}
% بسته‌ای برای تنطیم حاشیه‌های بالا، پایین، چپ و راست صفحه
\usepackage[top=30mm, bottom=30mm, left=25mm, right=30mm]{geometry}
% بسته‌‌ای برای ظاهر شدن شکل‌ها و تصاویر متن
\usepackage{graphicx}
\usepackage{color}
%بسته‌ای برای تنظیم فاصله عمودی خط‌های متن
\usepackage{setspace}
\usepackage{titletoc}
\usepackage{tocloft}
%با فعال کردن بسته زیر فوت‌نوت‌ها در هر صفحه ریست می‌شوند. حالت پیش‌فرض آن ریست شدن در هر فصل می‌باشد.
%\usepackage[perpage]{footmisc}
\usepackage{enumitem}
\usepackage{caption}

\usepackage{algorithm}
\usepackage{algorithmic}
% \usepackage{algpseudocode}
%\usepackage{titlesec}
% بسته‌ و دستوراتی برای ایجاد لینک‌های رنگی با امکان جهش
\usepackage[pagebackref=false,colorlinks,linkcolor=blue,citecolor=red]{hyperref}
\usepackage[nameinlink]{cleveref}%capitalize,,noabbrev
% \usepackage{polyglossia}

% \setdefaultlanguage[numerals=maghrib]{persian}
% \setotherlanguage{english}

 \AtBeginDocument{%
    \crefname{equation}{معادله}{equations}%
    \crefname{chapter}{فصل}{chapters}%
    \crefname{section}{بخش}{sections}%
    \crefname{appendix}{پیوست}{appendices}%
    \crefname{enumi}{مورد}{items}%
    \crefname{footnote}{زیرنویس}{footnotes}%
    \crefname{figure}{شکل}{figures}%
    \crefname{table}{جدول}{tables}%
    \crefname{theorem}{قضیه}{theorems}%
    \crefname{lemma}{لم}{lemmas}%
    \crefname{corollary}{نتیجه}{corollaries}%
    \crefname{proposition}{گزاره}{propositions}%
    \crefname{definition}{تعریف}{definitions}%
    \crefname{result}{نتیجه}{results}%
    \crefname{example}{مثال}{examples}%
    \crefname{remark}{نکته}{remarks}%
    \crefname{note}{یادداشت}{notes}%
}
% چنانچه قصد پرینت گرفتن نوشته خود را دارید، خط بالا را غیرفعال و  از دستور زیر استفاده کنید چون در صورت استفاده از دستور زیر‌‌، 
% لینک‌ها به رنگ سیاه ظاهر خواهند شد که برای پرینت گرفتن، مناسب‌تر است
%\usepackage[pagebackref=false]{hyperref}
% بسته‌ لازم برای تنظیم سربرگ‌ها
\usepackage{fancyhdr}
% بسته‌ای برای ظاهر شدن «مراجع»  در فهرست مطالب
\usepackage[nottoc]{tocbibind}
% دستورات مربوط به ایجاد نمایه
\usepackage{makeidx,multicol}
\setlength{\columnsep}{1.5cm}

\usepackage[bottom]{footmisc}
\usepackage{float}
%%%%%%%%%%%%%%%%%%%%%%%%%%
\usepackage{verbatim}
\makeindex
\usepackage{sectsty}
% فراخوانی بسته زی‌پرشین و تعریف قلم فارسی و انگلیسی
\usepackage{xepersian}%[extrafootnotefeatures]
\SepMark{-}
%حتماً از تک لایو 2014 استفاده کنید.
% \setmainfont[
%  UprightFont={XB Niloofar.ttf}
%  BoldFont={XB NiloofarBd.ttf}, 
% %  ItalicFont={XB NiloofarIt.ttf},
% %  BoldItalicFont={XB NiloofarBdIt.ttf}
%  ]{}
% HERE
\usepackage[T1]{fontenc}
% \setmainfont[Path=fonts/,
%         UprightFont=XB Niloofar.ttf,
%         ItalicFont=XB NiloofarIt.ttf]
%   {XB Niloofar.ttf}
\settextfont[Scale=1]{XB Niloofar.ttf}
% \setlatintextfont{Times New Roman.ttf}
\renewcommand{\labelitemi}{$\bullet$}
%%%%%%%%%%%%%%%%%%%%%%%%%%
% چنانچه می‌خواهید اعداد در فرمول‌ها، انگلیسی باشد، خط زیر را غیرفعال کنید.
%در غیر اینصورت حتماً فونت PGaramond را نصب کنید.
% \setdigitfont[Scale=1.1]{Yas.ttf}%%Yas
\setcounter{secnumdepth}{4}
%%%%%%%%%%%%%%%%%%%%%%%%%%
% تعریف قلم‌های فارسی اضافی برای استفاده در بعضی از قسمت‌های متن
% \defpersianfont\nastaliq[Scale=2]{IranNastaliq.ttf}
% \defpersianfont\chapternumber[Scale=3]{B Nazanin.ttf}
%\chapterfont{\centering}%
%%%%%%%%%%%%%%%%%%%%%%%%%%
% دستوری برای تغییر نام کلمه «اثبات» به «برهان»
\renewcommand\proofname{\textbf{برهان}}

% دستوری برای تغییر نام کلمه «کتاب‌نامه» به «منابع و مراجع«
\renewcommand{\bibname}{منابع و مراجع}


% Headings for every page of ToC, LoF and Lot
\setlength{\cftbeforetoctitleskip}{-1.2em}
\setlength{\cftbeforelottitleskip}{-1.2em}
\setlength{\cftbeforeloftitleskip}{-1.2em}
\setlength{\cftaftertoctitleskip}{-1em}
\setlength{\cftafterlottitleskip}{-1em}
\setlength{\cftafterloftitleskip}{-1em}
%%\makeatletter
%%%%\renewcommand{\l@chapter}{\@dottedtocline{1}{1em\bfseries}{1em}}
%%%%\renewcommand{\l@section}{\@dottedtocline{2}{2em}{2em}}
%%%%\renewcommand{\l@subsection}{\@dottedtocline{3}{3em}{3em}}
%%%%\renewcommand{\l@subsubsection}{\@dottedtocline{4}{4em}{4em}}
%%%%\makeatother


\newcommand\tocheading{\par عنوان\hfill صفحه \par}
\newcommand\loaheading{\par الگوریتم\hfill صفحه \par}
% \newcommand\khiarheading{\par معادله\hfill صفحه \par}

\newcommand\loeheading{\par معادله\hfill صفحه \par}

\newcommand\lofheading{\hspace*{.5cm}\figurename\hfill صفحه \par}
% \newcommand\lotheading{\hspace*{.5cm}\tablename\hfill صفحه \par}

% \newcommand\eqoheading{\hspace*{.5cm}\equationame\hfill صفحه \par}

\renewcommand{\cftchapleader}{\cftdotfill{\cftdotsep}}
\renewcommand{\cfttoctitlefont}{\hspace*{\fill}\LARGE\bfseries}%\Large
\renewcommand{\cftaftertoctitle}{\hspace*{\fill}}
\renewcommand{\cftlottitlefont}{\hspace*{\fill}\LARGE\bfseries}%\Large
\renewcommand{\cftafterlottitle}{\hspace*{\fill}}
\renewcommand{\cftloftitlefont}{\hspace*{\fill}\LARGE\bfseries}
\renewcommand{\cftafterloftitle}{\hspace*{\fill}}

%%%%%%%%%%%%%%%%%%%%%%%%%%
% تعریف و نحوه ظاهر شدن عنوان قضیه‌ها، تعریف‌ها، مثال‌ها و ...
%برای شماره گذاری سه تایی قضیه ها
\theoremstyle{definition}
\newtheorem{definition}{تعریف}[section]
\newtheorem{remark}[definition]{نکته}
\newtheorem{note}[definition]{یادداشت}
\newtheorem{example}[definition]{نمونه}
\newtheorem{question}[definition]{سوال}
\newtheorem{remember}[definition]{یاداوری}
\theoremstyle{theorem}{}
\newtheorem{theorem}[definition]{قضیه}
\newtheorem{lemma}[definition]{لم}
\newtheorem{proposition}[definition]{گزاره}
\newtheorem{corollary}[definition]{نتیجه}
%%%%%%%%%%%%%%%%%%%%%%%%
%%%%%%%%%%%%%%%%%%%
%%% برای شماره گذاری چهارتایی قضیه ها و ...
%%\newtheorem{definition1}[subsubsection]{تعریف}
%%\newtheorem{theorem1}[subsubsection]{قضیه}
%%\newtheorem{lemma1}[subsubsection]{لم}
%%\newtheorem{proposition1}[subsubsection]{گزاره}
%%\newtheorem{corollary1}[subsubsection]{نتیجه}
%%\newtheorem{remark1}[subsubsection]{نکته}
%%\newtheorem{example1}[subsubsection]{مثال}
%%\newtheorem{question1}[subsubsection]{سوال}

%%%%%%%%%%%%%%%%%%%%%%%%%%%%

% دستورهایی برای سفارشی کردن صفحات اول فصل‌ها
\makeatletter
\newcommand\mycustomraggedright{%
 \if@RTL\raggedleft%
 \else\raggedright%
 \fi}
\def\@makechapterhead#1{%
\thispagestyle{style1}
\vspace*{20\p@}%
{\parindent \z@ \mycustomraggedright
\ifnum \c@secnumdepth >\m@ne
\if@mainmatter

\bfseries{\Huge \@chapapp}\small\space \chapternumber {\huge\thechapter}
\par\nobreak
\vskip 0\p@
\fi
\fi
\interlinepenalty\@M 
\Huge \bfseries #1\par\nobreak
\vskip 120\p@

}

\thispagestyle{empty}
\newpage}
\bidi@patchcmd{\@makechapterhead}{\thechapter}{\tartibi{chapter}}{}{}
\bidi@patchcmd{\chaptermark}{\thechapter}{\tartibi{chapter}}{}{}
\makeatother

\pagestyle{fancy}
\renewcommand{\chaptermark}[1]{\markboth{\chaptername~\tartibi{chapter}: #1}{}}

\fancypagestyle{style1}{
\fancyhf{} 
\fancyfoot[c]{\thepage}
\fancyhead[R]{\leftmark}%
\renewcommand{\headrulewidth}{1.2pt}
}


\fancypagestyle{style2}{
\fancyhf{}
\fancyhead[R]{چکیده}
\fancyfoot[C]{\thepage{}}
\renewcommand{\headrulewidth}{1.2pt}
}

\fancypagestyle{style3}{%
  \fancyhf{}%
  \fancyhead[R]{فهرست نمادها}
  \fancyfoot[C]{\thepage}%
  \renewcommand{\headrulewidth}{1.2pt}%
}

\fancypagestyle{style4}{%
  \fancyhf{}%
  \fancyhead[R]{فهرست جداول}
  \fancyfoot[C]{\thepage}%
  \renewcommand{\headrulewidth}{1.2pt}%
}

\fancypagestyle{style5}{%
  \fancyhf{}%
  \fancyhead[R]{فهرست اشکال}
  \fancyfoot[C]{\thepage}%
  \renewcommand{\headrulewidth}{1.2pt}%
}

\fancypagestyle{style6}{%
  \fancyhf{}%
  \fancyhead[RO, LE]{فهرست مطالب}
  \fancyfoot[C]{\thepage}%
  \renewcommand{\headrulewidth}{1.2pt}%
}

\fancypagestyle{style7}{%
  \fancyhf{}%
  \fancyhead[R]{نمایه}
  \fancyfoot[C]{\thepage}%
  \renewcommand{\headrulewidth}{1.2pt}%
}

\fancypagestyle{style8}{%
  \fancyhf{}%
  \fancyhead[R]{منابع و مراجع}
  \fancyfoot[C]{\thepage}%
  \renewcommand{\headrulewidth}{1.2pt}%
}
\fancypagestyle{style9}{%
  \fancyhf{}%
  \fancyhead[R]{واژه‌نامه‌ی فارسی به انگلیسی}
  \fancyfoot[C]{\thepage}%
  \renewcommand{\headrulewidth}{1.2pt}%
}

\fancypagestyle{style10}{%
  \fancyhf{}%
  \fancyhead[R]{فهرست معادلات}
  \fancyfoot[C]{\thepage}%
  \renewcommand{\headrulewidth}{1.2pt}%
}

\fancypagestyle{style11}{%
  \fancyhf{}%
  \fancyhead[R]{فهرست الگوریتم‌ها}
  \fancyfoot[C]{\thepage}%
  \renewcommand{\headrulewidth}{1.2pt}%
}

%


%دستور حذف نام لیست تصاویر و لیست جداول از فهرست مطالب
\newcommand*{\BeginNoToc}{%
  \addtocontents{toc}{%
    \edef\protect\SavedTocDepth{\protect\the\protect\value{tocdepth}}%
  }%
  \addtocontents{toc}{%
    \protect\setcounter{tocdepth}{-10}%
  }%
}
\newcommand*{\EndNoToc}{%
  \addtocontents{toc}{%
    \protect\setcounter{tocdepth}{\protect\SavedTocDepth}%
  }%
}

\newcounter{savepage}
\renewcommand{\listfigurename}{فهرست اشکال}
\renewcommand{\listtablename}{فهرست جداول}
% \listofalgorithms

\newcommand{\listequationsname}{
\hfill
فهرست معادلات
\hfill
}
\newlistof{myequations}{equ}{\listequationsname}
\newcommand{\myequations}[1]{%
\addcontentsline{equ}{myequations}{\protect\numberline{\theequation}#1}\par}
\setlength{\cftmyequationsnumwidth}{2.5em} % Width of equation number in List of Equations

\newcommand{\listalgorithmsname}{
\hfill
فهرست الگوریتم‌ها
\hfill
}
\newlistof{myalgorithms}{algo}{\listalgorithmsname}
\newcommand{\myalgorithms}[1]{%
\addcontentsline{algo}{myalgorithms}{\protect\numberline{\thealgorithm}#1}\par}
\setlength{\cftmyequationsnumwidth}{2.5em} % Width of equation number in List of Equations
\renewcommand{\theequation}{\arabic{chapter}.\arabic{algorithm}}

%\renewcommand\cftsecleader{\cftdotfill{\cftdotsep}}
%%%%%%%%%%%%%%%%%%%%%%%%%%%%%
%%%%%%%%%%%%%%%%%%%%%%%%%%%%
% \numberwithin{section}{equation}
\renewcommand{\theequation}{\arabic{chapter}.\arabic{equation}}
\newcommand{\hst}[0]{\hspace{2mm}}
\newcommand{\hso}[0]{\hspace{1mm}}

\begin{document}
\baselineskip=.75cm
\linespread{1.75}
%% -!TEX root = AUTthesis.tex
% در این فایل، عنوان پایان‌نامه، مشخصات خود، متن تقدیمی‌، ستایش، سپاس‌گزاری و چکیده پایان‌نامه را به فارسی، وارد کنید.
% توجه داشته باشید که جدول حاوی مشخصات پروژه/پایان‌نامه/رساله و همچنین، مشخصات داخل آن، به طور خودکار، درج می‌شود.
%%%%%%%%%%%%%%%%%%%%%%%%%%%%%%%%%%%%
% دانشکده، آموزشکده و یا پژوهشکده  خود را وارد کنید
\faculty{دانشکده مهندسی کامپیوتر}
% گرایش و گروه آموزشی خود را وارد کنید
\department{}
% عنوان پایان‌نامه را وارد کنید
\fatitle{توليد موسيقی با محاسبات کوانتومی}
% نام استاد(ان) راهنما را وارد کنید
\firstsupervisor{دکتر مرتضی صاحب‌الزمانی}
%\secondsupervisor{استاد راهنمای دوم}
% نام استاد(دان) مشاور را وارد کنید. چنانچه استاد مشاور ندارید، دستور پایین را غیرفعال کنید.
%\firstadvisor{دکتر بهادر بخشی}
%\secondadvisor{استاد مشاور دوم}
% نام نویسنده را وارد کنید
\name{عرفان }
% نام خانوادگی نویسنده را وارد کنید
\surname{عابدی}
%%%%%%%%%%%%%%%%%%%%%%%%%%%%%%%%%%
\thesisdate{مهر ۱۴۰۰}

% چکیده پایان‌نامه را وارد کنید
\fa-abstract{
امروزه، با نزدیک‌شدن تکنولوژی‌های ساخت سخت‌افزار به محدودیت‌های فیزیکی قانون مور
% \fnote{Moore's law}
و پیش‌رفت روزافزون تکنولوژی‌های فوق سرد
% \fnote{Cryogenics}،
توجه بسیاری به دانشمندان به حوزه‌ی محاسبات کوانتومی معطوف شده است. 
کامپیوترهایی که در حال حاضر به صورت روزمره در حال استفاده هستند، در انجام محاسبات خود از قوانین فیزیک کلاسیک پیروی می‌کنند و به کامپیوترهای کلاسیک معروف هستند.
محاسباتی که توسط کامپیوترهای کوانتومی انجام می‌شوند، بر خلاف کامپیوترهای کلاسیک، تابع قوانین فیزیک کوانتومی هستند؛ به همین علت، بسیاری بر این باور هستند که این گونه کامپیوترها در آینده‌ای نزدیک، قادر به انجام محاسباتی خواهند بود که به سادگی توسط کامپیوترهای کلاسیک ممکن نیست.
یادگیری ماشین کوانتومی
% \fnote{Quantum Machine Learning}
به گروهی از الگوریتم‌های کوانتومی اطلاق می‌شود که همانند یادگیری ماشین کلاسیک، تعدادی پارامتر قابل تنظیم دارند که بهینه‌سازی این پارامترها برای رسیدن به خروجی مطلوب، توسط یک کامپیوتر کلاسیک انجام می‌گیرد.
در عین حال، از همان روزهای اولیه‌ی پیدایش کامپیوترها، بسیاری به دنبال تولید ملودی‌های موسیقی با استفاده از قدرت پردازشی کامپیوترها بوده‌اند و تاکنون، الگوریتم‌های متعددی برای حصول این امر پیشنهاد شده‌اند.
در این پروژه، امکان تولید موسیقی با استفاده از الگوریتم‌های یادگیری ماشین کوانتومی بررسی شده و برنامه‌ی نرم‌افزاری‌ای برای انجام این کار توسعه داده شده است.
% در این پروژه، سعی شده با به کار گیری از یادگیری ماشین کوانتومی، ملودی‌های موسیقی بدیعی تولید شود.
}


% کلمات کلیدی پایان‌نامه را وارد کنید
\keywords{
محاسبات کوانتومی، یادگیری ماشین، تولید موسیقی کامپیوتری
}



\AUTtitle
%%%%%%%%%%%%%%%%%%%%%%%%%%%%%%%%%%
\vspace*{7cm}
\thispagestyle{empty}
\begin{center}
\includegraphics[height=5cm,width=12cm]{figures/besm.jpg}
\end{center}
% تاییدیه دفاع
\newpage
\thispagestyle{empty}
%\fontsize{18pt}{19pt}\selectfont

\section*{صفحه فرم ارزیابی و تصویب پایان نامه- فرم تأیید اعضاء كميته دفاع}

\captionsetup[figure]{list=no}

\begin{figure}[H]
	\centering
	\includegraphics[scale=0.8]{taeid}
\end{figure}

\captionsetup[figure]{list=yes}

%%%%%%%%%%%%%%%%%%%%%%%%%%%%%%%%%%%%%%%%%%%%%%%%%%%%%%%%%%%%%%%%%%%%%%%%%%%%%%%%%%%%%%%%%%%%%%%%%%
%%%%%%%%%%%%%%%%%%%%%%%%%%%%%%%%%%%%%%%%%%%%%%%%%%%%%%%%%%%%%%%%%%%%%%%%%%%%%%%%%%%%%%%%%%%%%%%%%%
\newpage
\thispagestyle{empty}
\begin{picture}(50,50)
  \put(17,0){\includegraphics[scale=1.1]{fa-logo}}
  \put(4.5,-13){\footnotesize{دانشگاه صنعتی امیرکبیر}}
  \put(10.5,-27){\footnotesize{(پلی‌تکنیک تهران)}}
  \put(170,30){\bf{به نام خدا}}
  \put(140,-5){\Large\bf{تعهدنامه اصالت اثر}}
  \put(310,0){تاریخ: \datethesis}
\end{picture}

\vspace*{2.5cm}

اين‌جانب {\bf{\fname\lname}} متعهد می‌شوم که مطالب مندرج در این پایان‌نامه حاصل کار پژوهشی این‌جانب، تحت نظارت و راهنمایی اساتید دانشگاه صنعتی امیرکبیر بوده و به دست‌آوردهای دیگران که در این پژوهش از آن‌ها استفاده شده است،
مطابق مقررات و روال متعارف ارجاع و در فهرست منابع و مآخذ ذکر گردیده است. این پایان‌نامه قبلاً برای احراز هیچ مدرک هم‌سطح یا بالاتر ارائه نگردیده است.

در صورت اثبات تخلف در هر زمان، مدرک تحصیلی صادر شده توسط دانشگاه از درجه اعتبار ساقط بوده و دانشگاه حق پی‌گیری قانونی خواهد داشت.


کلیه نتایج و حقوق حاصل از این پایان‌نامه متعلق به دانشگاه صنعتی امیرکبیر می‌باشد. هرگونه استفاده از نتایج علمی و عملی، واگذاری اطلاعات به دیگران یا چاپ و تکثیر، نسخه‌برداری، ترجمه و اقتباس از این پایان نامه بدون موافقت کتبی دانشگاه صنعتی امیرکبیر ممنوع است. 
نقل مطالب با ذکر منبع بلامانع است.\\
\vspace{2.5cm}


{\centerline {\bf{\fname\lname}}}
\vspace*{.2cm}
{\centerline{امضا}}
%%%%%%%%%%%%%%%%%%%%%%%%%%%%%%%%%
% چنانچه مایل به چاپ صفحات «تقدیم»، «نیایش» و «سپاس‌گزاری» در خروجی نیستید، خط‌های زیر را با گذاشتن ٪  در ابتدای آنها غیرفعال کنید.
% پایان‌نامه خود را تقدیم کنید
% نیایش خود را در فایل زیر بنویسید.
%%\begin{acknowledgementpage}

\vspace{1.5cm}

{\nastaliq
{
 نويسنده پايان‌نامه، درصورت تمايل ميتواند برای سپاسگزاری پايان‌نامه خود را به شخص يا اشخاص و يا ارگان خاصی تقدیم نماید.
}}\end{acknowledgementpage}
\newpage
% سپاسگزاری را در فایل زیر بنویسید.
%%%%%%%%%%%%%%%%%%%%%%%%%%%%%%%%%%%%
\newpage\thispagestyle{empty}
% سپاس‌گزاری
% {\nastaliq
سپاس‌گزاری
% }
\\[2cm]

این‌جانب،
\fullname
،
از استاد راهنمای خود، جناب آقای 
\ffirstsupervisor \space
به خاطر تمامی کمک‌ها و راهنمایی‌های سازنده‌شان در مسیر انجام این پروژه و نوشتن این گزارش سپاس‌گزارم.














% با استفاده از دستور زیر، امضای شما، به طور خودکار، درج می‌شود.
\signature








%%%%%%%%%%%%%%%%%%%%%%%%%%%%%%%%%%%%%%%%%
%%%%%%%%%%%%%%%%%%%%%%%%%%%%%%%%%کدهای زیر را تغییر ندهید.
\newpage\clearpage

\pagestyle{style2}

\vspace*{-1cm}
\section*{\centering چکیده}
%\addcontentsline{toc}{chapter}{چکیده}
\vspace*{.5cm}
\ffa-abstract
\vspace*{2cm}


{\noindent\large\textbf{واژه‌های کلیدی:}}\par
\vspace*{.5cm}
\fkeywords
% دستور زیر برای شماره گذاری صفحات قبل از فصل اول با حروف ابجد است.
\pagenumbering{alph}
%-----------------------------------------------------------------------------
% فایل زیر دستورات مربوط به نمایش صفحات فهرست مطالب- فهرست اشکال و جداول است.
%{\pagestyle{style2}
%\tableofcontents}\newpage
%
%\listoffigures
\cleardoublepage
%\addtocontents{toc}{\tocheading}
\pagestyle{style6}
\tableofcontents
\pagestyle{style6}
\cleardoublepage
%اگر لیست تصاویر و لیست جداول ندارید ، کدهای زیر را با گذاشتن % در ابتدای آنها، غیرفعال کنید.
\BeginNoToc
%============
\addtocontents{lof}{\lofheading}% add heading to the first page in LoF
\pagestyle{style10}
\listofmyequations
% \listoffigures
\thispagestyle{style10}
\clearpage

% \addtocontents{lof}{\lofheading}% add heading to the first page in LoF
\pagestyle{style5}
\listoffigures
\thispagestyle{style5}
\clearpage
%============
%\addtocontents{lot}{\lotheading}% add heading to the first page in LoT
%\thispagestyle{style4}
%\listoftables
%\thispagestyle{style4}
%============
%\cleardoublepage
%
%\cleardoublepage
%\setcounter{savepage}{\arabic{page}}
%\mainmatter
%\addtocontents{toc}{\tocheading}% add heading to the first page in ToC, after frontmatter entries
\EndNoToc
% در صورت تمایل می‌توانید با فعال کردن دستور بالا، لیست تصاویر را به  پایان‌نامه خود اضافه کنید.
%-------------------------------------------------------------------------symbols(فهرست نمادها)
% وجود لیست نمادها الزامیست.(لطفاً نمادهای خود را جایگذین نمادهای پیش‌فرض کنید.)
%%%%%%%%%%%%%%

{\centering\LARGE\textbf{فهرست نمادها}\par}%

\pagenumbering{alph}
\setcounter{page}{\thesavepage}
%\setcounter{page}{6}
\vspace*{1cm}

\pagestyle{style3}
%\thispagestyle{empty}
%\addcontentsline{toc}{chapter}{فهرست نمادها}
\symb{\text{ نماد}}{مفهوم}
\\
%مقادیر بالا را تغییر ندهید
%%%%%%%%%%%%%%%%%%%%%%%%%%%%%%%%%%%%%%%%%%%%%%%%%%%%%%%%%
\symb{\mathbb{R}^n}{
فضای اقلیدسی با بعد $n$
}
\symb{\mathbb{S}^n}{
کره یکه $n$ بعدی
}
\symb{M^m}{
خمینه $m$-بعدی $M$
}
\symb{\mathfrak{X}(M)}{
جبر میدان‌های  برداری هموار روی $M$
}
\symb{\mathfrak{X}^1(M)}{
مجموعه میدان‌های برداری هموار یکه روی $(M,g)$ 
}
\symb{\Omega^p(M)}{
مجموعه $p$-فرمی‌های روی خمینه $M$
}
\symb{Q}{
اپراتور ریچی
}
\symb{\mathcal{R}}{
تانسور انحنای ریمان
}
\symb{ric}{
تانسور ریچی
}
\symb{L}{
مشتق لی
}
\symb{\Phi}{
2-فرم اساسی خمینه تماسی
}
\symb{\nabla}{
التصاق لوی-چویتای
}
\symb{\Delta}{
لاپلاسین ناهموار
}
\symb{\nabla^*}{
عملگر خودالحاق صوری القا شده از التصاق لوی-چویتای
}
\symb{g_s}{
متر ساساکی
}
\symb{\nabla}{
التصاق لوی-چویتای وابسته به متر ساساکی
}
\symb{\Delta}{
عملگر لاپلاس-بلترامی روی $p$-فرم‌ها
}

%%%%%%%%%%%%%%%%%%%%%%%%%%%%%%%%%%%%%%%

\thispagestyle{style3}
\newpage
%\pagestyle{style1}
%%%%%%%%%%%%%%%%%%%%%%%%%%%%%%%%%%%%


\pagenumbering{arabic}
\pagestyle{style1}
%--------------------------------------------------------------------------chapters(فصل ها)
\chapter{مقدمه}

% \section{محاسبات کوانتومی}

با گذشت چهل سال از طرح ایده‌ی محاسبات کوانتومی توسط ریچارد فاینمن
\cite{feynman}،
توجه جهان به کامپیوترهای کوانتومی در بالاترین سطح خود قرار دارد
عمده‌ی این توجه پس از ابداع الگوریتم فاکتورگیری شور
\fnote{Shor's factorization algorithm}
و الگوریتم جست‌وجوی گروور 
\fnote{Grover's quantum search algorithm}
در دهه‌ی نود میلادی آغاز شده است. از آن روز تاکنون، تلاش‌های بسیار زیادی برای طراحی الگوریتم‌های جدید برای کاربردهای متنوع و دسترسی به برتری محاسباتی کوانتومی در حال انجام هستند. در زمان تحریر این گزارش، شرکت‌های بسیاری از جمله آی‌بی‌ام، مایکروسافت و گوگل در حال سرمایه‌گذاری‌های جدی بر روی ساخت کامپیوترهای کوانتومی هستند. به عنوان مثال، در سال ۲۰۱۹،
شرکت گوگل آزمایشی 
\cite{google_supremacy}
انجام داد که نشان‌گر برتری محاسباتی کوانتومی بود، به این معنا که پیچیدگی زمانی انجام این آزمایش در حالت کلاسیک بسیار بزرگ‌تر از حالت کوانتومی است.

با این وجود، به علت محدودیت تعداد کیوبیت\footnote{معادل کوانتومی یک بیت کلاسیک}های کامپیوترهای کوانتومی فعلی، تلاش‌های اندکی در راستای تولید قطعات موسیقی با استفاده از الگوریتم‌های کوانتومی صورت گرفته است. دو الگوریتمی که تاکنون در این راستا تحت بررسی قرار گرفته‌اند، تولید موسیقی با استفاده از تولیدکننده‌ی اعداد تصادفی
\fnote{Random number generator}
و گشت کوانتومی روی گراف
\fnote{Quantum walk on graph}
\cite{miranda}
بوده‌اند، اما به نظر می‌رسد به علت وجود معادلات موجی در فیزیک کوانتومی و این حقیقت که موسیقی در اصل تشکیل شده از موج‌های صوتی است، در آینده می‌توان ارتباطات خیلی بیش‌تری بین تولید موسیقی و محاسبات کوانتومی کشف کرد.

% \newpage
\section{هدف پروژه}
% موضوع این پایان‌نامه، در گام نخست بررسی شیوه‌های موجود استفاده از الگوریتم‌های کوانتومی برای تولید موسیقی؛ و در گام دوم، معرفی دو شیوه جدید برای تولید موسیقی با استفاده از یادگیری ماشین کوانتومی است.
هدف این پروژه، طراحی نرم‌افزاری است که بتواند با استفاده از الگوریتم‌های موجود یادگیری ماشین کوانتومی و یک  مجموعه داده از برخی موسیقی‌های موجود، اقدام به تولید قطعه‌ی موسیقی
\fnote{Musical piece}
جدیدی کند.
مجموعه داده‌ی این پروژه به صورت سری گسسته‌ای از نت‌ها و آکوردهای موسیقی است که به صورت یک فایل 
\lr{midi}
ذخیره شده‌اند.
عمده‌ی نرم‌افزارهای پخش محتوای چندرسانه‌ای، قابلیت پخش موسیقی با استفاده از فایل‌های 
\lr{midi}
به عنوان ورودی را دارند، اما
 شایان ذکر است که این نوع ذخیره‌سازی، با ذخیره‌سازی‌های شناخته‌شده‌تر محتوای صوتی مانند فایل‌های 
 \lr{mp3}
 متفاوت است؛ چراکه فایل‌های
 \lr{mp3}
 با استفاده از پردازش سیگنال و تبدیل فوریه اقدام به ذخیره‌سازی می‌کنند.
 
\newpage
\section{اجزاء پروژه} \label{sec:parts}
پیاده‌سازی این پروژه به‌صورت کلی شامل ۳ بخش می‌شود:
\begin{enumerate}
    \item ماژول
    \lr{Midi}
    که مسئولیت پردازش داده‌های موسیقی را بر عهده دارد. این ماژول تعدادی فایل که متشکل از چندین نت و آکورد است را به صورتی که به عنوان ورودی برای الگوریتم کوانتومی به راحتی قابل استفاده باشد در می‌آورد. هم‌چنین، این ماژول پس از دریافت خروجی تولید شده به وسیله‌ی اجرای الگوریتم یادگیری ماشین کوانتومی بر روی ورودی‌ها، فایل قطعه‌ی موسیقی جدیدی را تولید می‌کند.
    \item ماژول
    \lr{QLSTM}
    که با استفاده از الگوریتم حافظه‌ی طولانی کوتاه مدت کوانتومی
    \fnote{Quantum long short-term memory}
    \cite{chen_qlstm}،
    اقدام به تولید قطعات موسیقی می‌کند.
    
    \item ماژول
    \lr{QuGAN}
    که با استفاده از الگوریتم شبکه‌ی زایای دشمن‌گونه کوانتومی
    \fnote{Quantum generative adversarial network}
    \cite{lloyd_qugan} \cite{zoufal_qugan}،
    اقدام به تولید قطعات موسیقی می‌کند.

\end{enumerate}

\section{ابزارهای مورد استفاده برای پیاده‌سازی}

برای پیاده‌سازی اجزای مختلف این پروژه، از چند کتابخانه‌ی نرم‌افزاری استفاده شده که در فصول بعدی نقش آن‌ها به صورت کامل شرح داده خواهد شد، اما در این بخش صرفا به عنوان مقدمه از آن‌ها نام برده می‌شود:
\lr{
\begin{itemize}
\itemsep-0.5em 
	\item \lr{Music21}
	\item \lr{PyTorch}
	\item \lr{PennyLane}
\end{itemize}
}

\vspace{-1.7cm}
\section{ساختار پایان‌نامه}
این پایان‌نامه در پنج فصل به بررسی موضوع مطرح‌شده می‌پردازد. در فصل دوم مفاهیم اولیه‌ی یادگیری ماشین، محاسبات کوانتومی و تئوری موسیقی تشریح می‌شود. در فصل سوم مروری بر تلاش‌هایی که تا پیش از این در راستای حصول این امر صورت گرفته شده انجام می‌شود.
در فصل چهارم شیوه‌های نوین پیشنهاد شده توسط این پایان‌نامه مطرح می‌شود و در نهایت، در فصل پنجم و آخر به نتیجه‌گیری و مسیرهای پیشنهادی برای پژوهش‌های آتی پرداخته می‌شود.
\chapter{مفاهیم پایه}
در این فصل به معرفی مفاهیم پایه موردنیاز برای درک قسمت‌های مختلف پروژه و جزییات پیاده‌سازی آن پرداخته می‌شود.

\newcommand{\dd}[1]{\mathrm{d}#1}

% Quantum Computation

\section{محاسبات کوانتومی}

\subsection{سیستم‌های تک‌کیوبیتی}

طبق قوانین فیزیک کوانتومی، حالت یک سیستم می‌تواند به صورت ترکیب خطی‌ای از چند حالت پایه باشد، که این حالت‌های پایه، حالت‌هایی هستند که در قوانین فیزیک کلاسیک نیز حالات صحیحی برای توصیف سیستم هستند.
در محاسبات کوانتومی از بردارهای عمودی برای نشان‌دادن وضعیت یک سیستم استفاده می‌شود و وضعیت سیستم‌های چندکیوبیتی را نیز می‌توان از روی بردارهای یک سیستم تک‌کیوبیتی نیز ساخت.
\\
به طور معمول، در محاسبات کوانتومی، صرفا سیستم‌هایی که دارای تنها دو حالت پایه هستند بررسی می‌شوند، به همین علت، بردارهای فضای حالات سیستم‌های تک‌کیوبیتی ۲-بعدی هستند. لذا دو بردار مستقل یکه به عنوان بردارهای پایه‌ی این فضا تعیین داده می‌شوند که رابطه‌ی ۱-به-۱ ای با حالات پایه‌ی بیت‌های کلاسیک دارند.
این حالات پایه به صورت زیر تعیین می‌شوند:
\begin{equation}
    |0\rangle = \begin{bmatrix} 1 \\ 0 \end{bmatrix} 
    \mspace{18mu}
    |1\rangle = \begin{bmatrix} 0 \\ 1 \end{bmatrix}
\end{equation} 
\myequations{بردارهای پایه تک‌کیوبیتی}

نشان دادن بردارهای حالت به صورت 
\lr{$|0\rangle$} و \lr{$|1\rangle$}
به نمادگذاری \lr{bra-ket} دیراک
\fnote{Dirac's bra-ket notation}
معروف است و به ازای هر
\lr{ket}
به صورت زیر:
\begin{equation}
    |\psi\rangle = \alpha |0\rangle + \beta |1\rangle
    = \begin{bmatrix}
    \alpha \\[3pt]
    \beta
    \end{bmatrix}
\end{equation}
یک \lr{bra}
به این صورت تعریف می‌شود:
\begin{equation}
    \langle \psi| = \alpha^* \langle0| + \beta^* \langle1| = \begin{bmatrix} \alpha^* & \beta^* \end{bmatrix} 
\end{equation}

که در این‌جا نماد \lr{$\alpha^*$}
به معنای مزدوج مخلط 
\fnote{Complex conjugate}
عدد \lr{$\alpha$}
است.
\newpage

حالت‌های سیستم در فیزیک کوانتومی در اکثر اوقات به صورت بردارهای مختلط بهنجار
\fnote{Normalizable}
نشان داده می‌شوند.
\begin{equation}
    |\psi\rangle = \alpha |0\rangle + \beta |1\rangle = \begin{bmatrix} \alpha \\ \beta \end{bmatrix} 
    ; \mspace{18mu}
    \alpha, \beta \in \mathbb{C}
\end{equation}
% \newpage
\myequations{فرم کلی بردار وضعیت سیستم تک کیوبیتی}
که به امر ایجاد یک بردار وضعیت از ترکیب خطی دو بردار وضعیت دیگر، اصل برهم‌نهی کوانتومی
\fnote{Quantum Superposition}
گفته می‌شود.

بهنجار بودن به معنای صدق شرایط زیر است:
\begin{equation}
    \alpha\alpha^* + \beta\beta^* = |\alpha|^2 + |\beta|^2 = 1
\end{equation}
\myequations{شرط بهنجار بودن}
% \newpage
\begin{figure}
	\centering
	\includegraphics[scale=0.4]{figures/bloch.jpg}
	\caption{کره‌ی بلاخ}
	\label{fig:bloch}
\end{figure}
به علت شرط بهنجاری، می‌توان حالت کلی یک سیستم تک‌کیوبیتی را به صورت زیر نیز نوشت:
\begin{equation}
|\psi\rangle = \begin{bmatrix} e^{i\phi_1}\cos{\tfrac{\theta}{2}} \\[6pt] e^{i\phi_2}\sin{\tfrac{\theta}{2}} \end{bmatrix} 
\mspace{18mu}
\theta, \phi_1, \phi_2 \in {\rm I\!R}
\end{equation}
\myequations{حالت معادل فرم کلی بردار تک‌کیوبیتی}

که آن را می‌توان به صورت زیر نیز نوشت:
\begin{equation}
|\psi\rangle = e^{i\phi_1} \begin{bmatrix} \cos{\tfrac{\theta}{2}} \\[6pt] e^{i\phi_2-\phi_1}\sin{\tfrac{\theta}{2}} \end{bmatrix} = e^{i\phi_1} \begin{bmatrix} \cos{\tfrac{\theta}{2}} \\[6pt] e^{i\phi}\sin{\tfrac{\theta}{2}} \end{bmatrix}
\Rightarrow |\psi\rangle = \begin{bmatrix} \cos{\tfrac{\theta}{2}} \\[6pt] e^{i\phi}\sin{\tfrac{\theta}{2}} \end{bmatrix}
\end{equation}
\myequations{مهم نبودن فاز کلی سیستم}
در این معادله،
\lr{$\phi_1$}
 فاز کلی سیستم نامیده می‌شود که طبق قوانین فیزیک کوانتومی، در رفتار سیستم فاقد اهمیت است و به همین علت در مرحله‌ی آخر از آن صرف نظر شده‌است. \\
در نهایت، حالت کلی یک کیوبیت را می‌توان با استفاده از ابزاری به نام کره‌ی بلاخ (شکل
\ref{fig:bloch})
نمایش داد.

\subsection{سیستم‌های چندکیوبیتی}
دو سیستم تک‌کیوبیتی جداگانه را می‌توان به طور مجزا و به صورت زیر نمایش داد:
\begin{equation}
|a\rangle = \begin{bmatrix} a_0 \\ a_1 \end{bmatrix}, \quad |b\rangle = \begin{bmatrix} b_0 \\ b_1 \end{bmatrix}
\end{equation}
\myequations{دو کیوبیت مجزا}
در عین حال، می‌توانیم بردار وضعیت آن‌ها را به صورت هم‌زمان با استفاده از عملگری به نام ضرب تانسوری
\fnote{Tensor product}
تعریف کنیم که به صورت زیر عمل می‌کند:
\begin{equation}
|b\rangle \otimes |a\rangle = \begin{bmatrix} b_0 \times \begin{bmatrix} a_0 \\ a_1 \end{bmatrix} \\[12pt] b_1 \times \begin{bmatrix} a_0 \\ a_1 \end{bmatrix} \end{bmatrix} = \begin{bmatrix} b_0 a_0 \\[2pt] b_0 a_1 \\[2pt] b_1 a_0 \\[2pt] b_1 a_1 \end{bmatrix} = |ba\rangle
\end{equation}
\myequations{ضرب تانسوری}

\subsection{درهم‌تنیدگی}

درهم‌تنیدگی کوانتومی\fnote{Quantum Entanglement}
، یکی از اصول فیزیک کوانتومی است و به این معناست که برخی بردار وضعیت‌های سیستم‌های چندکیوبیتی را نمی‌توان به صورت ضرب تانسوری دو بردار تک‌کیوبیتی مجزا تعریف کرد. این امر نشان‌گر این است که وضعیت این دو کیوبیت به هم وابسته هستند.
به عنوان مثال، اگر بردارهای زیر که در محاسبات کوانتومی به وضعیت‌های بل 
\fnote{Bell states}
معروف هستند را در نظر بگیریم:
\begin{equation}
|{\Phi_\pm}\rangle = \frac{1}{\sqrt 2}\big(|0\rangle| 0\rangle\pm |1\rangle| 1\rangle\big), \qquad |{\Psi_{\pm}}\rangle=\frac{1}{\sqrt 2}\big(|0\rangle| 1\rangle\pm  |1\rangle|0\rangle\big)
\end{equation}
\myequations{بردار وضعیت‌های بل}

مشاهده می‌کنیم که هیچ‌کدام از این بردارها را نمی‌توان به صورت ضرب تانسوری‌ای از ترکیب خطی بردارهای
\lr{$|0\rangle$} و \lr{$|1\rangle$}
نوشت.

\subsection{گیت‌های کوانتومی}
در کامپیوترهای کلاسیک، محاسبات با استفاده از گیت‌هایی همانند 
\lr{AND}، \lr{OR} و NOT
انجام می‌شود.
معادل این گیت‌ها در محاسبات کوانتومی، گیت‌های کوانتومی هستند. این گیت‌ها به فرم ماتریس‌های یکانی 
\fnote{Unitary matrix}
\lr{$2^n \times 2^n$}
هستند که در این‌جا، عدد \lr{$n$}
نشان‌گر تعداد کیوبیت‌های سیستم است.

\subsubsection{
    گیت‌های کوانتومی تک‌کیوبیتی
}
در این بخش، صرفا تعدادی از گیت‌های کوانتومی به صورت خلاصه معرفی می‌شوند و اثر آن‌ها بر روی پایه‌های برداری فضای سیستم‌های تک‌کیوبیتی نشان داده می‌شود؛ چراکه تاثیر این گیت‌ها بر بردار وضعیت کیوبیت‌های دل‌خواه، با استفاده از ترکیب خطی تاثیر این گیت‌ها بر پایه‌های برداری به دست می‌آید.
\begin{equation}
X = \begin{bmatrix} 0 & 1 \\ 1 & 0 \end{bmatrix} \qquad
Y = \begin{bmatrix} 0 & -i \\ i & 0 \end{bmatrix} \qquad
Z = \begin{bmatrix} 1 & 0 \\ 0 & -1 \end{bmatrix} \qquad
H = \frac{1}{\sqrt{2}} \begin{bmatrix} 1 & 1 \\ 1 & -1 \end{bmatrix}
\end{equation}
\myequations{گیت‌های پائولی و هادامارد}
تمامی گیت‌های تک‌کیوبیتی، حالت خاصی از گیت پارامتردار 
\lr{$U_3$}
هستند.

\begin{equation}
U_3(\theta, \phi, \lambda) = \begin{bmatrix} \cos(\frac{\theta}{2}) & -e^{i\lambda}\sin(\frac{\theta}{2}) \\[6pt]
            e^{i\phi}\sin(\frac{\theta}{2}) & e^{i(\phi+\lambda)}\cos(\frac{\theta}{2})
     \end{bmatrix}
\end{equation}
\myequations{گیت \lr{$U_3$}}

\subsubsection{
    گیت‌های کوانتومی چند‌کیوبیتی
}
گیت‌های چندکیوبیتی نیز، همانند بردارهای وضعیت سیستم‌های چندکیوبیتی، دو نوع متفاوت دارند. در این بخش -برای سادگی محاسبات- تنها گیت‌های دوکیوبیتی را بررسی می‌کنیم؛ اما همین روابط برای تعداد کیوبیت‌های بالاتر نیز صادق است. \\
نوع اول، گیت‌هایی هستند که می‌توان آن‌ها را به صورت ضرب تانسوری دو گیت تک‌کیوبیتی تجزیه کرد.
\\
به عنوان مثال:
\begin{equation}
    \begin{bmatrix}
    0 & 1 & 0 & 0 \\[3pt]
    1 & 0 & 0 & 0 \\[3pt]
    0 & 0 & 0 & 1 \\[3pt]
    0 & 0 & 1 & 0 
    \end{bmatrix} =
    \begin{bmatrix}
    1 & 0 \\[3pt]
    0 & 1 
    \end{bmatrix} \otimes
    \begin{bmatrix}
    0 & 1 \\[3pt]
    1 & 0
    \end{bmatrix}
    = \mathbb{I} \otimes X
\end{equation}
\myequations{گیت چندکیوبیتی ترکیبی}
این گیت معادل این امر است که هم‌زمان یک گیت همانی یا
$\mathbb{I}$
بر روی کیوبیت اول و یک گیت
$X$
بر روی کیوبیت دوم اعمال شود.

نوع دوم، گیت‌هایی هستند که به ضرب تانسوری دو گیت تک‌کیوبیتی تجزیه‌پذیر نیستند و تنها همین نوع گیت‌ها هستند که در هنگام اعمال بر روی برخی از حالت‌های کیوبیتی برهم‌نهیده، منجر به ایجاد درهم‌تنیدگی می‌شوند، به عنوان مثال، گیت
$CNOT$
\fnote{Controlled NOT}
به این صورت تعریف می‌شود:
\begin{equation}
    CNOT = \begin{bmatrix}
    1 & 0 & 0 & 0 \\[3pt]
    0 & 1 & 0 & 0 \\[3pt]
    0 & 0 & 0 & 1 \\[3pt]
    0 & 0 & 1 & 0 
    \end{bmatrix}
\end{equation}
\myequations{گیت CNOT}

به عنوان یک مثال از ایجاد درهم‌تنیدگی و برهم‌نهی، می‌توان دید:
\begin{equation}
    CNOT(H\otimes\bbmath{I}(|00\rangle)) = CNOT(\frac{1}{\sqrt{2}} \big( |0\rangle + |1\rangle \big) \otimes |0\rangle ) = \frac{1}{\sqrt{2}} (|00\rangle + |11\rangle)
\end{equation}
\myequations{تاثیر گیت \lr{CNOT} در ایجاد درهم‌تنیدگی}
این گیت به این دلیل نام‌گذاری شده که تاثیر آن بر روی کیوبیت دوم، توسط وضعیت کیوبیت اول کنترل شده؛ به این معنا که تنها در صورتی که کیوبیت اول در وضعیت
$|1\rangle$
باشد، گیت 
$X$
بر روی کیوبیت دوم اعمال خواهد شد.

\subsection{اندازه‌گیری}
اندازه‌گیری در محاسبات کوانتومی را می‌توان به گونه‌های مختلفی تعریف کرد، در این متن، یکی از این شیوه‌ها به عنوان معیار در نظر گرفته شده و تنها به آن پرداخته می‌شود.
عمل اندازه‌گیری در فیزیک کوانتومی، یک بردار وضعیت (که ممکن است برهم‌نهیده باشد) را ورودی گرفته و یک عدد حقیقی بین
$0$
و
$1$
را خروجی می‌دهد.
این اندازه‌گیری‌ها با توجه به یک مشاهده‌پذیر 
\fnote{Observable}
انجام می‌گیرند. مشاهده‌پذیرها در فیزیک کوانتومی، ماتریس‌های هرمیتی
\fnote{Hermitian matrix}
هستند. در این متن، فرض می‌شود که همیشه اندازه‌گیری با توجه به مشاهده‌پذیر 
$Z$
انجام می‌شود و به صورت زیر تعریف می‌شود:
\begin{equation}
    \langle \psi| Z^{\otimes n} | \psi\rangle
\end{equation}
\myequations{اندازه‌گیری کوانتومی}
که
$n$
تعداد کیوبیت‌های سیستم است. به عنوان مثال:
\begin{equation}
    \langle 1 | Z | 1 \rangle = 
    \begin{bmatrix}
    0 & 1
    \end{bmatrix} 
    \begin{bmatrix}
    1 & 0 \\[3pt]
    0 & -1 \\[3pt]
    \end{bmatrix}
    \begin{bmatrix}
    0 \\[3pt] 1
    \end{bmatrix} 
    = \begin{bmatrix}
    0 & 1
    \end{bmatrix} 
    \begin{bmatrix}
    0 \\[3pt] -1
    \end{bmatrix}
    = -1
\end{equation}
\myequations{مثال اندازه‌گیری کوانتومی}
که معادل قرار گرفتن وضعیت کیوبیت بعد از اندازه‌گیری در حالت 
$|1\rangle$
خواهد بود. در صورتی که بردار موردنظر دچار برهم‌نهی باشد، خروجی اندازه‌گیری را امیدریاضی اندازه‌گیری‌های متعدد در نظر گرفته می‌شود..

در صورتی که این اندازه‌گیری‌ها به صورت جداگانه بر روی کیوبیت‌های سیستم اعمال شوند و در سیستم در‌هم‌تنیدگی وجود داشته‌باشد؛ درایه‌های آرایه‌ای که از این اندازه‌گیری‌ها به وجود می‌آید به میزان درهم‌تنیدگی موجود در سیستم با هم مرتبط خواهند بود. به عنوان مثال، اگر در سیستم
$\frac{1}{\sqrt{2}} (|00\rangle + |11\rangle)$
کیوبیت اول اندازه‌گیری شود و بعد از اندازه‌گیری در حالت 
$|0\rangle$
قرار بگیرد، کیوبیت دوم نیز حتما در حالت
$|0\rangle$
خواهد بود و بالعکس.


\newpage

\section{یادگیری ماشین}
در این بخش تنها به مفاهیمی از یادگیری ماشین که برای فهم فصول بعدی لازم هستند اشاره می‌شود. \\
در حالت کلی، برای تعریف الگوریتم یا مدل‌های یادگیری ماشین، فرض می‌شود یک مجموعه داده موجود است که داده‌های آن به صورت دسته‌هایی دوتایی به شکل زیر هستند:
\begin{equation}
    (x_i, \hso y_i) \hst ; \hst y_i = g(x_i) \quad  \forall i : 1 \leq i \leq k
\end{equation}
\myequations{نحوه‌ی بررسی مجموعه داده‌ها در یادگیری ماشین}

که
$k$
نشان‌گر تعداد داده‌های موجود در مجموعه داده است؛
به این معنا که فرض می‌شود بین 
$x$ها
و
$y$ها
رابطه‌ی ریاضی‌ای وجود دارد و هدف از طراحی مدل، کشف همین رابطه است. \\
می‌توان الگوریتم یا مدل‌های یادگیری ماشین را به شکل تابعی به صورت زیر نمایش داد:
\begin{equation}
    \Hat{y}_i = f(x_i, \hso \theta) \hst ; \hst x_i \in \mathbb{R}^m, \hspace{1.2mm} \theta \in \mathbb{R}^n
\end{equation}
\myequations{نمایش ریاضی مدل یادگیری ماشین}
تابع
$f$
دو نوع ورودی دریافت می‌کند، یکی
$x$
که معادل داده‌هایی‌ست که از یک مجموعه داده تعیین شده استخراج می‌شود و دیگری
$\theta$
که مجموعه‌ای از پارامترهایی تنظیم‌پذیر است. در مرحله‌ی تمرین دادن
\fnote{Training phase}
این مدل، سعی می‌شود با هدف کمینه کردن یک تابع هزینه
\fnote{Cost function}
،بهینه‌سازی این پارامترها صورت گیرد تا خروجی الگوریتم به جواب دل‌خواه نزدیک‌تر شود.
به این بهینه‌سازی، مرحله‌ی تمرین
تابع هزینه نیز عمدتا با این نیت تعریف می‌گردد که ملاک خوبی از رضایت‌بخشی خروجی الگوریتم باشد. به عنوان مثال، اگر در یک مجموعه داده،
$x$
معادل مجموعه‌ای از اعداد در یک سری عددی
و
$y$
معادل عدد بعدی ظاهرشده در این مجموعه اعداد باشد، در هنگام تعریف یک مدل یادگیری ماشین برای پیدا کردن عدد بعدی یک مجموعه با گرفتن اعداد قبلی آن، می‌توان تابع هزینه در هر مرحله از تمرین را به این صورت تعریف کرد:
\begin{equation}
    \mathcal{L}_f = \frac{1}{k} \sum_{i=1}^{k} (\hat{y}_i - y)^2
\end{equation}
\myequations{تابع خطای میانگین مربعات}
که به این نوع تابع هزینه، تابع خطای میانگین مربعات
\fnote{Mean squared error}
گفته می‌شود.
\newpage

یهینه‌سازی پارامترها در الگوریتم‌های یادگیری ماشین، به طور معمول توسط الگوریتم کاهش گرادیانی
\fnote{Gradient descent}
انجام می‌شود؛ به این معنا که میزان تغییرات پارامترها در زمان، طبق معادلات زیر به تابع هزینه وابسته می‌شود.

\begin{equation}
    \frac{\partial \theta_j(t)}{\partial t} = - \eta \frac{\partial \mathcal{L}_f}{\partial \theta_j}
    = - \eta \sum_i \frac{\partial f \big(x_i, \theta (t)\big)}{\partial \theta_j} \frac{\partial \mathcal{L}_f}{\partial f \big(x_i, \theta(t)\big)}
\end{equation}
\myequations{الگوریتم کاهش گرادیانی}
در معادله‌ی بالا، متغیر
$\eta$
به نرخ یادگیری 
\fnote{Learning rate}
معروف است و میزان تغییرات پارامترها با توجه به گرادیان تابع هزینه را تنظیم می‌کند.


\subsection{
شبکه‌های عصبی بازگشتی
\protect\fnote{Recurrent neural networks}
}

\begin{figure}
	\centering
	\includegraphics[scale=0.35]{figures/architecture-rnn.png}
	% To make sure citation doesn't appear in list of figures
	\caption [
	ساختار شبکه‌های عصبی بازگشتی
	]{
	ساختار شبکه‌های عصبی بازگشتی 
	\cite{rnncheat}
	}
	\label{fig:rnnarch}
\end{figure}

شبکه‌های عصبی بازگشتی
گونه‌ی خاصی از الگوریتم‌های یادگیری ماشین هستند که ساختار کلی آن‌ها در شکل
\ref{fig:rnnarch}
آمده است.
در این‌گونه شبکه‌های عصبی نیز
$x$
ها نشان‌گر داده‌های ورودی از مجموعه داده‌ها و
$y$
ها نشان‌گر خروجی‌های الگوریتم هستند.
پارامترهای تنظیم‌پذیر در واحد‌های آبی‌رنگ که به 
واحد‌های بازگشتی
\fnote{Recurrent block}
معروف هستند قرار می‌گیرند.
نکته‌ی اصلی شبکه‌های عصبی بازگشتی این است که هر واحد بازگشتی، در هنگام انجام محاسبات، از نتایج محاسبات واحد‌های محاسباتی قبل از خود استفاده می‌کند، چراکه در این صورت می‌تواند با کسب آگاهی از خروجی‌های گذشته و ترتیب آن‌ها، خروجی معنی‌داری تولید کند.
این‌گونه الگوریتم‌ها اغلب در یادگیری ویژگی‌های داده‌های ترتیبی
\fnote{Sequential data}
کاربرد دارند؛ به این معنا که نه تنها خود خروجی‌های الگوریتم، بلکه ترتیب آن‌ها هم از اهمیت بالایی برخوردار است.
نت‌های موسیقی را نیز می‌توان به صورت مجموعه‌ای از داده‌های ترتیبی در نظر گرفت، چراکه هر مجموعه‌ای از نت‌های موسیقی، صدایی آهنگین تولید نمی‌کند و ترتیب نت‌های قرار گرفته در یک ملودی نیز برای این‌که توسط گوش انسان به عنوان یک موسیقی حقیقی در نظر گرفته شوند حائز اهمیت است.

\subsubsection{حافظه‌ی طولانی کوتاه-مدت}
حافظه‌ی طولانی کوتاه-مدت
\fnote{Long short-term memory}
نوع خاصی از شبکه‌های عصبی بازگشتی است که ساختار کلی  واحد بازگشتی آن در شکل
\ref{fig:lstmblock}
آمده است.
یکی از ویژگی‌های مهمی که حافظه‌های طولانی کوتاه-مدت را در مقایسه با باقی شبکه‌های عصبی بازگشتی متمایز می‌کند، این است که به جای انتقال یک واحد اطلاعات از هر مرحله به مرحله‌ی بعدی، دو واحد اطلاعات را منتقل می‌کند.

\begin{figure}
	\centering
	\includegraphics[scale=0.4]{figures/lstm-rec-unit.png}
	% To make sure citation doesn't appear in list of figures
	\caption [
	واحد بازگشتی حافظه‌ی طولانی کوتاه-مدت
	]{
	واحد بازگشتی حافظه‌ی طولانی کوتاه-مدت 
	\cite{rnncheat}
	}
	\label{fig:lstmblock}
\end{figure}

گیت‌های بازگشتی، واحدهای پردازشی‌ای هستند که به طور معمول در شبکه‌های عصبی بازگشتی حضور دارند و با علامت 
$\Gamma$
نشان داده می‌شوند. حالت کلی این گیت‌ها به صورت زیر است:
\begin{equation}
    \Gamma = \sigma(Wx^{(t)} + Ua^{(t-1)} + b)
\end{equation}
\myequations{گیت‌های بازگشتی}
که در معادله‌ی بالا، پارامترهای
$W, U, b \hspace{0.5mm}$
همان پارامترهای تنظیم‌پذیر الگوریتم هستند و
$\sigma$
یک تابع غیرخطی است که برای تعمیم توانایی مدل‌سازی شبکه‌های عصبی استفاده می‌شود و معمولا تابع فعال‌سازی
\fnote{Activation function}
نامیده می‌شود.
در حافظه‌های کوتاه بلند-مدت،، معمولا از تابع سیگموید
\fnote{Sigmoid function}
به عنوان تابع فعال‌سازی استفاده می‌شود که تعریف این تابع به صورت زیر است:
\begin{equation}
    S(x) = \frac{1}{1+e^{-x}} = \frac{e^x}{e^x + 1}
\end{equation}
\myequations{تابع سیگموید}

همان‌طور که در شکل
\ref{fig:lstmblock}
مشاهده می‌شود، هر واحد بازگشتی حافظه‌های طولانی کوتاه-مدت شامل چهار گیت بازگشتی است که هر کدام به منظور خاصی تعبیه شده‌اند و سعی در پیاده‌سازی رفتار خاصی را دارند.

\begin{itemize}
    \item
    گیت به‌روزرسانی
    \fnote{Update gate}
    یا 
    $\Gamma_u$
    که میزان حفظ اطلاعات مراحل گذشته در محاسبات فعلی را تعیین می‌کند.
    
    \item
    گیت ارتباط
    \fnote{Relevance gate}
    یا 
    $\Gamma_r$
    که میزان پاک‌شدن اطلاعات مراحل گذشته در محاسبات فعلی را تعیین می‌کند.
    
    \item
    گیت خروجی
    \fnote{Output gate}
    یا 
    $\Gamma_o$
    که میزان حفظ شدن اطلاعات محاسبات فعلی برای انتقال به مرحله‌ی بعدی را تعیین می‌کند.
    
    \item
    گیت فراموشی
    \fnote{Forget gate}
    یا
    $\Gamma_f$
    که میزان پاک‌شدن اطلاعات محاسبات فعلی برای انتقال به مرحله‌ی بعدی را تعیین می‌کند.
\end{itemize}

شایان ذکر است که پارامترهای این گیت‌های بازگشتی، معمولا پارامترهای مستقلی هستند و لذا به صورت جداگانه نیز بهینه‌سازی می‌شوند.

در نهایت، خروجی‌های واحد
$t$
-ام یک
حافظه‌ی کوتاه بلند-مدت که با
$c^{(t)}$
و
$a^{(t)}$
نشان داده می‌شوند،
به صورت زیر محاسبه می‌شوند:
\begin{equation}
\begin{gathered}
    \hat{c}^{(t)} = tanh(W_c[\Gamma_r * a^{(t-1)}, x^{(t)}] + b_c) \\
    c^{(t)} = \Gamma_u * \hat{c}^{(t)} + \Gamma_f * c^{(t-1)} \\
    a^{(t)} = \Gamma_o * c^{(t)}
\end{gathered}
\end{equation}
\myequations{خروجی‌های واحد بازگشتی حافظه‌ی طولانی کوتاه-مدت}

\newpage

\subsection{
شبکه‌های زایای دشمن‌گونه
\protect\fnote{Generative adversarial networks}
}
هر شبکه‌ی زایای دشمن گونه، متشکل از دو مدل یادگیری ماشین است. در هنگام مراحل یادگیری، این دو مدل با یک‌دیگر رقابت می‌کنند و سعی می‌کنند دیگری را در یک بازی مجموع-صفر
\fnote{Zero-sum game}
شکست دهند. یکی از این مدل‌ها، مدل زایا
\fnote{Generative model}
و مدل دیگر، مدل فرق‌گذار
\fnote{Discriminative model}
نام دارد. 
با فرض این‌که مجموعه داده‌ای با توزیعی
\fnote{Distribution}
مشخص موجود باشد، مدل زایا سعی می‌کند داده‌های جدیدی تولید کند که شباهت زیادی به داده‌های توزیع واقعی تولید کند. در عین حال، مدل فرق‌گذار، سعی می‌کند پس از گرفتن یک داده‌ی ورودی، تشخیص دهد که این داده متعلق به آن توزیع است یا خیر.
\\
برای واضح‌تر شدن چگونگی کارکرد شبکه‌های زایای دشمن‌گونه، می‌توان به مساله‌ی تولید پرتره اشاره کرد؛ به این معنا که مجموعه داده‌ای از عکس‌های پرتره‌ی صورت انسان‌های متفاوتی موجود است. در این حالت، مدل زایا تلاش می‌کند تا عکس پرتره‌ی جدیدی تولید کند و مدل فرق‌گذار با گرفتن ورودی‌ای، سعی می‌کند تشخیص دهد که این ورودی توسط مدل زایا تولید شده یا از مجموعه داده‌ی اصلی نمونه‌برداری شده.
در نهایت، در صورت موفق بودن تمرین این دو مدل، مدل زایا می‌تواند عکس‌های پرتره‌ی جدیدی تولید کند که کامپیوتری بودن آن‌ها حتی توسط خود انسان‌ها هم ممکن نباشد و مدل فرق‌گذار می‌تواند عکس‌های کامپیوتری را از عکس‌های واقعی به خوبی تشخیص دهد. 
% \end{equation}

\begin{figure}
	\centering
	\includegraphics[scale=0.2]{figures/fakeperson.jpg}
	% To make sure citation doesn't appear in list of figures
	\caption [
	نمونه عکس پرتره‌ی تولید شده توسط شبکه‌ی زایای دشمن‌گونه
	]{
	نمونه عکس پرتره‌ی تولید شده توسط شبکه‌ی زایای دشمن‌گونه
	\cite{thisperson}
	}
	\label{fig:lstmblock}
\end{figure}
% \chapter{کارهای پیشین}

\section{الگوریتم‌های کلاسیک تولید موسیقی}

از همان ابتدای پیدایش کامپیوترهای امروزی، موسیقی‌دان‌ها در تلاش برای تولید ملودی‌های جدید به وسیله‌ی کامپیوترها بوده‌اند.
مثال‌هایی از تلاش‌ها برای تولید موسیقی به وسیله‌ی محاسبات کلاسیک در زیر آمده‌اند:

\begin{itemize}
    \item 
    در دهه‌ی ۴۰ میلادی، پژوهشگران مجمع علمی-صنعتی استرالیا\fnote{Australian council for scientific and industrial research (CSIR)} بلندگویی را به یک کامپیوتر ام‌کا۱\fnote{MK1} متصل کردند تا صدایی که کامپیوتر در حین اجرای برنامه‌ها تولید می‌کرد را بشنوند. در همین راستا و در سال ۱۹۵۱، جف هیل\fnote{Geoff Hill} که ریاضی‌دانی با پیش‌زمینه‌ای در موسیقی بود، برنامه‌ای بر روی این کامپیوتر اجرا کرد تا صدای تولید شده تا حد بسیار خوبی شبیه ملودی‌های موسیقی شود
\cite{CSIR_music}.

\item
مدل‌های ترجمه‌ای\fnote{Translational models} سعی می‌کنند تا داده‌های غیرصوتی را به صوت تبدیل کنند. به عنوان مثال، اگر مدلی ساخته شود که سعی کند عکس‌های با فرمت 
\lr{jpg}
را به فایل‌هایی با فرمت
\lr{mp3}
تبدیل کند (چرا که هر دوی این فرمت‌ها از تبدیل فوریه برای کدگذاری داده‌ها استفاده می‌کنند)، ممکن است عکس یک خط صاف را به عنوان یک موسیقی با یک نت ثابت تعبیر کند.

\item
مدل‌های گرامری\fnote{Grammatical models} سعی می‌کنند با پردازش کردن داده‌های موسیقیایی به عنوان جملات زبانی که همانند زبان طبیعی، از قواعد گرامری خاصی تبعیت می‌کند، الگوریتم‌های پردازش زبان طبیعی\fnote{Natural Language Processing (NLP)} را بر روی داده‌های موسیقیایی اعمال کنند.

\item
متدهای تکاملی، با استفاده از چهارچوب الگوریتم‌های ژنتیک\fnote{Genetic algorithms}، با یک سری از نت‌های موسیقی به عنوان ژن افراد یک جمعیت برخورد می‌کنند و بعد از ترکیب کردن ژن‌های این افراد و ایجاد جهش‌های تصادفی در این ژن‌ها، موسیقی‌های جدیدی تولید کنند.
به عنوان مثال، یک الگوریتم ژنتیک ممکن است دو گام زیر را به این صورت ترکیب کند:
\begin{equation}
\begin{gathered}
    C_{major} = [C, D, E, F, G, A, B, C]\\[3pt]
    A_{major} = [A, B, C\sharp, D, E, F\sharp, G\sharp, A] \\
    Genetic(A_{major}, C_{major}) = [C, D, E, F, E, F\sharp, G\sharp, C] 
\end{gathered}
\end{equation}
\myequations{خروجی الگوریتم ژنتیک بر روی دو گام موسیقی}
که نیمه‌ی اول گام 
$C_{major}$
با نیمه‌ی دوم گام
$A_{major}$
ترکیب شده‌است و نت آخر از 
$A$
به
$C$
جهش یافته‌است.

\item 
با پیدایش و همه‌گیر شدن حوزه‌های هوش مصنوعی و یادگیری ماشین، مدل‌های یادگیری ماشین زیادی برای تولید موسیقی پیشنهاد شده است.
کتابخانه‌ی 
\lr{Magenta}
که توسط بخشی از تیم کتاب‌خانه‌ی یادگیری ماشین
\lr{TensorFlow}
توسعه یافته‌است، شامل پیاده‌سازی انواع مختلفی از این مدل‌ها، از جمله
\lr{Melody RNN}
\cite{magenta_melodyrnn}
و
\lr{GANSynth}
\cite{magenta_gansynth}
است. مدل اول با استفاده از نوع خاصی از حافظه‌های طولانی کوتاه-مدت
و نوع دوم با استفاده از نوع خاصی از شبکه‌های زایای دشمن‌گونه اقدام به تولید موسیقی‌های بدیع می‌کند.
\end{itemize}

\section{الگوریتم‌های کوانتومی تولید موسیقی}

در سال‌های اخیر و با همه‌گیرتر شدن حوزه‌های کوانتومی، چندین پیشنهاد برای بررسی ارتباط بین این حوزه و موسیقی پیشنهاد شده‌است. منبع
\cite{Putz_quantum_music}
برای اولین بار، بررسی تولید موسیقی کوانتومی\fnote{Quantum music} و به‌طور کلی‌تر، هنرهای کوانتومی\fnote{Quantum arts} را مطرح کرد؛ ایده‌ی اصلی این مقاله این است که در صورتی که بتوان در دنیای روزمره، موج‌های کوانتومی‌ای تولید کرد که بتوانند از نویزهای محیط در امان بمانند، می‌توان موج کوانتومی‌ای تولید کرد که در وضعیت برهم‌نهی از دو موسیقی متفاوت باشد. در صورت تحقق این امر، در هنگام شنیده‌شدن این موج توسط گوش انسان، عمل اندازه‌گیری انجام می‌گیرد و به همین خاطر، هر شنونده موسیقی متفاوتی می‌شنود.
این مقاله، منجر به پروژه‌ی
\lr{Quantum Music}
\cite{quantum_music_event}
شد که مقالات منتشر شده در راستای این پروژه، ایده‌های الهام گرفتن از معادلات موج کوانتومی برای تولید موسیقی
\cite{Helweg_QInspired_Music}
و استفاده از تصادفی‌بودن نتایج برخی اندازه‌گیری‌ها در محاسبات کوانتومی و بازپخت کوانتومی\fnote{Quantum annealing} برای تولید موسیقی
\cite{Kirke_QC_Music}
را به صورت ابتدایی بررسی می‌کنند.

در نهایت، منبع
\cite{miranda}
روش استفاده از گشت کوانتومی\fnote{Quantum walk} روی گراف را
برای تولید موسیقی پیشنهاد می‌کند.
مساله‌ی گشت کوانتومی روی گراف، معادل کوانتومی مساله‌ی گشت گراف کلاسیک است؛ به این معنا که یک موجود ریاضیاتی به نام گردشگر\fnote{Walker} در یکی از گره‌های یک گراف قرار می‌گیرد، به صورتی که حرکت این موجود در گراف، تابع قوانین خاصی است. هدف این گونه مسائل، آنالیز مسیرهای طی شده توسط این موجود و توزیع احتمالاتی\fnote{Probability distribution} مکان نهایی آن در گراف بعد از گذشت زمان مشخصی است.
اما به خاطر تفاوت‌های موجود بین این دو حوزه، نتایج آن‌ها نیز متفاوت است
\cite{Kempe_qwalk}.
به عنوان مثال، رابطه‌ی انحراف معیار\fnote{Standard deviation} ($\sigma$)
توزیع احتمالاتی گشت با تعداد گام‌های طی‌شده
($T$)
در حالات کلاسیک و کوانتوم به شکل زیر متفاوت است:
\begin{equation}
\begin{gathered}
    \sigma^2_{Classical} \sim T \\[3pt]
    \sigma^2_{Quantum} \sim T^2
\end{gathered}
\end{equation}

\begin{figure}
	\centering
	\includegraphics[scale=0.5]{figures/classical_distribution.png}
	\includegraphics[scale=0.5]{figures/quantum_distribution.png}
	% To make sure citation doesn't appear in list of figures
	\caption [
	مقایسه‌ی توزیع احتمالی گشت گرافی کلاسیک و کوانتوم
	]{
	مقایسه‌ی توزیع احتمالی گشت گرافی کلاسیک (راست) و کوانتومی (چپ)
	\cite{cirq_qwalk}
	}
	\label{fig:walk_distr}
\end{figure}

\myequations{انحراف معیار گشت‌های کوانتوم و کلاسیک}
به طور خلاصه، الگوریتم گشت گراف کوانتومی، وضعیت گردشگر را در چند کیوبیت کدگذاری می‌کند و با فرض شروع مسیر گردشگر از یک گره خاص در گراف، در هر مرحله، با استفاده از یک کیوبیت سکه\fnote{Coin qubit} که در هر مرحله تحت تاثیر یک گیت هادامارد قرار می‌گیرد، وضعیت کیوبیت‌ها را به برهم‌نهی‌ای از وضعیت ندهای مجاور آن گراف می‌برد. نکته‌ای که باعث تفاوت حرکات گردشگر کلاسیک و کوانتومی می‌شود، اثر تداخل کوانتومی\fnote{Quantum interference} است که احتمالات حضور گردشگر در برخی نت‌ها را تقویت و در برخی دیگر، تضعیف می‌کند.
مقایسه‌ی توزیع احتمال مکان نهایی گردشگر در اجرای الگوریتم گشت بر روی گراف حلقوی‌ای با شش گره در حالت کلاسیک و در حالت کوانتومی‌ای که کیوبیت سکه‌ی آن در زمان
$t=0$
از وضعیت اولیه‌ی زیر شروع شده:
\begin{equation}
    |i\rangle = \frac{|0\rangle + i|1\rangle}{\sqrt{2}}
\end{equation}
\myequations{وضعیت اولیه‌ی کیوبیت‌ها در یک گشت کوانتومی}
در شکل
\ref{fig:walk_distr}
آمده است.
بیان ریاضی عملگرهای الگوریتم گشت کوانتومی به صورت زیر است:
\begin{equation}
\begin{gathered}
    |q_t\rangle = |state_t\rangle \otimes |coin_t\rangle \\[3pt]
    U = |0\rangle \langle 0| \otimes \sum_j |j + 1\rangle \langle j| + |1\rangle \langle 1| \otimes \sum_j |j - 1\rangle
    \langle j| \\[3pt]
    S = U(H \otimes \mathbb{I}) \\[3pt]
    |q_{t+1}\rangle = S|q_t \rangle
\end{gathered}
\end{equation}
\myequations{بیان ریاضی عمل‌گرهای الگوریتم گشت کوانتومی}

که منظور از بردارهایی به شکل
$|j\rangle$
و 
$|j+1\rangle$
نمایش وضعیت کیوبیت‌های بردارهای وضعیت به شکل عدد طبیعی است که عمل‌گرهای جمع و تفریق هم‌نهشت\fnote{Modular arithmetic} با پیمانه‌ی
$2^n$
(که در این‌جا برابر با 
$2^3 = 8$
است)
به آن‌ها اعمال می‌شود. به عنوان مثال:
\begin{equation}
    \begin{gathered}
        |7\rangle = |111\rangle \\[3pt]
        |7 + 1\rangle = |8 \hso mod \hso 8\rangle = |0\rangle = |000\rangle \\[3pt]
        |0 - 1\rangle = |-1 \hso mod \hso 8\rangle = |7\rangle = |111\rangle
    \end{gathered}
\end{equation}
\myequations{مثال عمل‌گرهای جمع و تفریق حلقوی در الگوریتم گشت کوانتومی}

الگوریتم پیشنهاد شده در 
\cite{miranda}
نمونه‌گیری از یک الگوریتم گشت کوانتومی روی یک گراف با هشت گره است که تنها به سه کیوبیت نیاز دارد. این الگوریتم پس از اندازه‌گیری وضعیت گردشگر گراف در انتهای چند مرحله گشت، وضعیت
$|000\rangle$
را به عنوان سکوت و هفت وضعیت پایه‌ی دیگر را به عنوان هفت نت ماژور (معادل کلیدهای سفید پیانو) در نظر می‌گیرد و بر اساس چندین بار اجرای این الگوریتم، قطعه‌ای در یک کلید ماژور تولید می‌کند.
% \chapter{پیاده‌سازی و نتایج نو}

در این پروژه، برای (تا جایی که نویسنده اطلاع دارد) اولین بار از الگوریتم‌های یادگیری ماشین کوانتومی برای تولید موسیقی استفاده شده و همان‌طور که در بخش
\ref{sec:parts}
اشاره شد، شامل سه ماژول 
\lr{Midi, QLSTM} 
و
\lr{QuGAN}
است که در این فصل توضیحات کامل عمل‌کرد آن‌ها شرح داده می‌شود.

\section{ماژول 
\lr{Midi}
}
مجموعه داده‌ی این پروژه، شامل ۹۲ قطعه‌ی موسیقی پیانو به صورت فایل‌های 
\lr{midi}
است، هرکدام از این فایل‌ها، مجموعه‌ای از نت‌ها، آکوردها و زمان پخش آن نت/آکورد از ابتدای قطعه به میلی‌ثانیه است.
این ماژول، تنها یک‌بار پوشه‌ی شامل مجموعه داده‌ها را به صورت کامل بررسی کرده و نت‌های تمامی قطعات را به صورت متوالی در یک لیست ذخیره کرده و در نهایت در یک فایل واحد با نام
\lr{notes.pk}
ذخیره می‌کند. شایان ذکر است که در صورت برخورد با یک آکورد، نت‌های آن را استخراج کرده و همانند چند نت عادی که پشت سر هم پخش می‌شوند، آن‌ها را به لیست نت‌ها اضافه می‌کند.
این ماژول سپس در هر بار اجرای کد، با گرفتن پارامتری به نام
\lr{SequenceLength}
تعداد زیادی جفت ورودی و خروجی برای مدل یادگیری ماشین کوانتومی فراهم می‌کند؛ به این صورت:

\begin{equation}
    \begin{gathered}
    input_i = [i, ..., SequenceLength + i] \\[3pt]
    output_i = [SequenceLength + i + 1] \\[3pt]
    where \hso 0 \leq i \leq n - SequenceLength - 1 \hso ; \hso n = \#notes
    \end{gathered}
\end{equation}
\myequations{نحوه تولید ورودی و خروجی ماژول \lr{Midi}}

\section{ماژول
\lr{QLSTM}
} 
این ماژول بر اساس حافظه‌های طولانی کوتاه-مدت کوانتومی که در بخش
\ref{sec:qlstm}
معرفی شدند طراحی شده است.
در کتاب‌خانه‌ی 
\lr{PyTorch}
هر مدل تمرین‌پذیر یادگیری ماشین، زیرکلاسی از کلاس
\lr{torch.nn.Module}
است. به همین دلیل، هر واحد بازگشتی کوانتومی این الگوریتم در کلاس
\lr{QLSTMCell}
تعریف شده که زیرکلاسی از کلاس 
\lr{torch.nn.Module}
است.
ورودی‌های مهمی که برای ساختن نمونه‌ای از این کلاس لازم است به شرح زیر هستند:
\begin{itemize}
    \item 
    \lr{n\_qubits}
    که تعداد کیوبیت‌های استفاده شده در مدارهای کوانتومی این واحد بازگشتی است.
    \item
    \lr{n\_qlayers}
    که تعداد لایه‌های موجود در قسمت پارامتردار مدارهای کوانتومی را تعیین می‌کند.
    \item
    \lr{input\_size}
    که ابعاد داده‌های ورودی به واحد بازگشتی را تعیین می‌کند.
    \item
    \lr{hidden\_size}
    که ابعاد داده‌ی خروجی واحد بازگشتی را تعیین می‌کند.
\end{itemize}
% \chapter{چالش‌های انجام پروژه}

با این‌که چالش‌های متعددی در مسیر انجام این پروژه خودنمایی کردند، در این فصل صرفا به برخی از چالش‌های مهم‌تر خواهیم پرداخت که می‌توان آن‌ها را در سه دسته‌ی فرآیندی، معماری و محدودیت‌های چارچوبی دسته‌بندی کرد.

\section{چالش‌های فرآیندی}

\subsection{ترکیب فرآیندهای ظاهر و باطن سامانه}

یکی از مهم‌ترین چالش‌های فرآیندی که در روال توسعه اسکرام در این پروژه مشخص شد، لزوم جداسازی \lr{Backlog}های ظاهر\footnote{\lr{Front-end}} و باطن\footnote{\lr{Back-end}} سامانه از یکدیگر بود. با توجه به اینکه در این پروژه از سامانه جیرا برای مدیریت فرآیند توسعه استفاده می‌شود، در ابتدا برای هر استوری کاربر، دو زیرمجموعه فرانت-اند و بک-اند تعریف شده بود که در عمل مشخص شد این نوع تعریف، کارآیی لازم را ندارد و در فرآیندهای چابک، باعث توسعه ناقص چند اسپرینت و درنهایت تکمیل تعدادی از آن‌ها بصورت یکجا بجای تکمیل هرکدام در ترتیب درست می‌شود.\\

البته این مسئله با رعایت دقیق تقدم‌ها و تاخرها تاحدی قابل پیش‌بینی است، اما نکته مهم در این‌جا، چابک بودن روال توسعه است (در مقابل برنامه‌محوری\footnote{\lr{Plan-driven}})؛ پس تقدم و تاخر پیش‌بینی شده لزوما محقق نخواهد شد و اگر در اواسط کار، نیازمندی‌ها تغییرکنند و یا زوایای پنهانی از سامانه شناسایی شود، حتما این مورد می‌تواند مشکل‌ساز باشد. بنابراین اگر روال توسعه سنتی بود و از یک برنامه دقیق و غیرقابل انعطاف استفاده می‌شد شاید هرگز چنین چالشی بوجود نمی‌آمد، اما در آن شرایط کل روال اسکرام زیرسوال می‌رفت و دیگر کاربردی نداشت!\\

\newpage

\subsection{ترکیب فرآیندهای نقش‌های مختلف کاربران}

یک چالش دیگر، مرکب بودن انواع نقش‌های کاربران در یک اسپرینت\footnote{\lr{Sprint}} بود. برای مثال در استوری‌های تعریف شده برای اسپرینت دوم از فرآیند توسعه، هم برای نقش فروشنده استوری کاربر تعریف شده و هم برای نقش خریدار. با توجه به این‌که برای بسیاری از بخش‌ها، حتما باید یک نقش تا حد مشخصی تکمیل شود و سپس توسعه نقش دیگر آغاز شود، این تقسیم‌بندی نیز می‌تواند باعث توسعه ناقص ولی همزمان چند اسپرینت شود. بنابراین، تنها توجه به تقدم و تاخر پیاده‌سازی‌ها کافی نیست و باید تا جای ممکن از ترکیب نقش‌های مختلف کاربران در یک اسپرینت پرهیز شود (مگر این‌که از لحاظ منطقی مشکلی برای تقدم و تاخرهای پیش‌بینی شده ایجاد نکند).\\

\subsection{تولید داده‌های آزمون}

یکی دیگر از چالش‌ها، لزوم مشخص کردن آزمون‌های سطح سیستم (و نه صرفا آزمون‌های واحد) در \lr{Backlog}هاست. با توجه به اینکه برای آزمون انسانی و دستی هر بخش از این سامانه به حجم قابل توجهی از داده‌های تست نیاز داریم، تولید این حجم از داده‌های آزمون بصورت دستی امکان‌پذیر نیست؛ بنابراین لازم است تا بخشی تحت عنوان آزمون‌های سطح سیستم از همان ابتدای توسعه شکل بگیرد و پیش‌اجراهایی برای آزمون‌های سیستمی از ابتدا نوشته شود که در آن‌ها، تعداد مناسبی از داده‌های آزمون بطور خودکار تولید شده و برای آزمون‌های انسانی آتی و یا حتی خود آزمون‌های سیستمی خودکار مورد استفاده قرار بگیرند.\\

چالش گفته شده را نباید با روال طبیعی آزمون‌های خودکار اشتباه گرفت. مجددا تاکید می‌شود که داده‌های تولید شده برای آزمون را نمی‌توان بصورت دستی و انسانی در این حجم ایجاد کرد و حتما بایستی بصورت خودکار ساخته شوند؛ اما استفاده از این داده‌ها لزوما خودکار نخواهد بود. بنابراین نه تنها آزمون خودکار، بلکه آزمون اولیه دستی نیز به این داده‌ها برای تحقق خود نیاز خواهد داشت.\\

\newpage

\subsection{تمرکز بیشتر بر طراحی دنباله درخواست‌ها}

اگر این پروژه بخواهد در یک محیط صنعتی و با شرایط زمانی گسترده‌تر از حالت فعلی انجام بپذیرد، حتما باید مدت زمان و \lr{Backlog}های بیشتری به بحث طراحی، بخصوص طراحی روند ارتباط فرانت-اند و بک-اند و دنباله درخواست‌ها و پاسخ‌های ارسالی و دریافتی برای انجام عملیات‌های اصلی سامانه اختصاص یابد. (منظور از روند ارتباط، معرفی نقاط دسترسی نهایی (یا همان اندپوینت\footnote{\lr{Endpoint}}‌ها) و جزئیات آن‌ها نیست. بلکه هر عملکرد معنایی در این سامانه، شامل دنباله‌ای از درخواست‌های شبکه می‌شود که به ترتیب پشت سر هم اجرا شده و ممکن است هر یک برای اجرای خود از داده‌های خروجی قبلی استفاده نماید.)\\

برای مثال، در اواخر فصل قبلی درمورد روال دو مرحله‌ای ورود و ثبت‌نام کاربران در سامانه توضیح داده شد. این روال درواقع یکی از همین دنباله‌هاست که مشخص بودن آن حتی پیش از طراحی مدل‌داده‌ها (و یا همزمان با آن) می‌تواند باعث تغییرات بنیادی در طراحی آن شود. همچنین اگر این طراحی‌ها مشخص باشد می‌توان به بخشی از چالش اول گفته شده (ترکیب فرآیندهای ظاهر و باطن سامانه) نیز غلبه کرد؛ زیرا درآن‌صورت، فعالیت‌هایی که یکی از ظواهر سامانه(مثلا نرم‌افزار کاربردی تلفن هوشمند) برای انجام یک عملیات خاص باید انجام دهد و دنباله درخواست‌هایی که باید به باطن سامانه (که همان سامانه اصلی است) مشخص خواهد بود و تقدم و تاخر در طراحی هرکدام نسبت به یکدیگر مستقل خواهد بود.

\section{چالش‌های معماری}

\subsection{انتقال تغییرات داده به لایه‌های بیرونی در ری‌اکت}

یکی از مهم‌ترین چالش‌هایی که با معماری انتقال داده‌ها درون نرم‌افزار کاربردی این سامانه (که با ری‌اکت پیاده‌سازی شده است) وجود داشت،‌ مسئله بازتاب تغییرات داده‌ها در لایه‌های بیرونی نرم‌افزار (و صفحات قبلی و عمومی‌تر) و درادامه، انتقال تغییرات به سامانه اصلی بود. به این‌صورت که اگر از صفحه لیست فروشگاه‌ها وارد صفحه یک فروشگاه شده، و از لیست محصولات آن وارد صفحه تغییر اطلاعات یک محصول شویم و اطلاعات آن را تغییر بدهیم، اطلاعات وارد شده باید درهنگام برگشت از صفحه تغییرات، در لیست محصولات اعمال شده باشد.

با معماری که در ابتدا استفاده شده بود تغییرات بلافاصله در لیست محصولات اعمال نمی‌شد و حتما باید یکبار لیست محصولات بسته و دوباره باز می‌شد. در واقع‌ معماری استفاده شده برای ذخیره داده‌های فروشگاهی در ابتدا، برای تغییر داده‌های مذکور، لیست جدید تمام داده‌های زیرمجموعه فروشگاه‌های یک فروشنده را در ورودی دریافت می‌کرد و کل آن‌ها را بازنویسی می‌کرد. شاید فقط یک خصوصیت مختصر از یکی از محصولات تغییر کرده بود، اما برای تغییر همان یک خصوصیت باید تمامی فروشگاه‌های کاربر بروزرسانی می‌شدند.

چون درون صفحات داخلی، به اطلاعات صفحات بیرونی دسترسی کاملی وجود ندارد،‌ عملا ایجاد این دسترسی سربار زیادی برای ساختار طبیعی یک نرم‌افزار تحت چارچوب ری‌اکت نیتیو ایجاد می‌کرد. علاوه بر این، خصوصیاتی از مدل داده‌های مربوط به فروشگاه‌های یک فروشنده بودند که عملا توسط نرم‌افزار کاربردی دریافت نمی‌شدند و صرفا توسط سامانه اصلی مدیریت شده ولی در پایگاه‌داده موجود بودند؛ پس عملا بازنویسی کل این لیست توسط نرم‌افزار کاربردی سامانه، باعث پاک شدن تمام آن اطلاعات می‌شد!

بنابراین، این شیوه حتما باید تغییر می‌کرد. برای حل این چالش، برای هرکدام از موجودیت‌های مرتبط با یک فروشنده (از جمله فروشگاه، محصول و کوپن تخفیف) یک گره\footnote{\lr{Node}} جداگانه در گرف‌کیوال تعریف شد که فقط خصوصیات اولیه\footnote{\lr{Primitive}} آن موجودیت را تغییر می‌داد و تنظیم می‌کرد. برای خصوصیاتی که لیست و آرایه بودند، مانند محصولات یک فروشگاه، سه گره جداگانه ایجاد شد که هرکدام یکی از سه وظیفه ایجاد\footnote{\lr{Create}}، ویرایش\footnote{\lr{Edit}} و حذف\footnote{\lr{Delete}} یک عضو از آن آرایه را انجام می‌دادند.

\begin{figure}[H]
	\centering
	\includegraphics[scale=0.55]{rnch}
	\caption{گره‌های تعریف شده در گرف‌کیوال برای ویرایش آرایه محصولات از یک فروشنده}
	\label{fig:rnch}
\end{figure}

برای گره‌های ویرایش و حذف، حتما باید مشخص می‌شد که منظور ما کدام عضو از آن موجودیت است. مثلا برای حذف یک محصول از درون لیستی از محصولات یک فروشگاه، باید مشخص باشد آن محصول کدام عضو آن لیست است. چه برای فروشگاه‌ها و چه برای محصولات، فرض شده که هیچ دو محصولی در یک فروشگاه، و هیچ دو فروشگاهی برای یک فروشنده وجود ندارند که نام و دسته‌بندی آن‌ها یکسان باشد. بنابراین از لحاظ مفهومی، زوج خصوصیت \{نام، دسته‌بندی\} بعنوان یک کلید اولیه\footnote{\lr{Primary‌ Key}} از هر محصول یا فروشگاه انتخاب شدند. در \cref{fig:rnch} نیز برای گره \lr{removeStore} مطلب اخیر قابل مشاهده است.

از طرفی دیگر درون نرم‌افزار پیاده‌شده با ری‌اکت نیتیو، روش مرسوم ایجاد تغییرات در داده‌ها از صفحات داخلی‌تر به صفحات بیرونی‌تر است. اما باتوجه به مشکلات ذکر شده در این پروژه از این روش استفاده نشده؛ بلکه تغییرات درهیچ یک از صفحات بصورت محلی در ابتدا اعمال نمی‌شوند و از طریق بیرونی‌ترین صفحه به سامانه اصلی (از طریق درخواست گرف‌گیوال) ارسال می‌شوند. اگر سامانه اصلی آن را تایید کند، این استاندارد رعایت شده که سامانه اصلی همواره شی یک سطح بالاتر از نقطه تغییرات را (پس از اعمال تغییرات جدید) بازگرداند. بنابرین صفحه بیرونی این شی جدید را دریافت کرده و سپس اقدام به تغییر کل وضعیت\footnote{\lr{State (useState Hook)}} داخلی از خودش می‌کند و این تغییر با استفاده از یک محتوا\footnote{\lr{Context (useContext Hook)}} در ریکعت نیتیو، به اطلاع تمامی صفحات داخلی‌تر می‌رسد و آن‌ها نیز داده جدید را دریافت می‌کنند\cite{react:hooks}.

\subsection{تنظیم سطح شفافیت و دسترسی در گرف‌کیوال}

در مدل نوشته برای گرف‌کیوال، تمام داده‌ها از لحاظ شفافیت و سطوح دسترسی، یکسان و در یک سطح دیده می‌شوند. اما برای مثال، شماره تلفن همراه برخی از کاربران نباید برای کاربران دیگر قابل دریافت باشد! از طرفی،‌ اگر این محدودیت دسترسی را بخواهیم در نرم‌افزار کاربردی سامانه پیاده‌سازی کنیم ساده‌انگارانه خواهد بود و از لحاظ امنیت اطلاعات راهکار پذیرفته شده‌ای نیست؛ درواقع می‌توان گفت این موضوع لازم است اما نه کافی.

همواره باید درنظر داشت که سامانه اصلی می‌تواند توسط هر کاربری فراخوانی شود و لزوما کاربر آن نرم‌افزار کاربردی تلفن هوشمند و یا پنل تحت وب مدیران نخواهد بود. پس این محدودیت‌ها باید صریحا از سمت سامانه اصلی مدیریت و اعمال شوند.

در توابع پیاده‌سازی‌کننده گره‌های گرف‌کیوال، شی‌ای تحت عنوان \lr{Context} در ورودی داده می‌شود که می‌تواند شامل اطلاعات احرازهویت و حساب کاربر اجراکننده درخواست فعلی باشد. از روی این شی می‌توان هویت و سطح دسترسی هر کاربر را درحین درخواست هر یک از گره‌های گرف‌کیوال تعیین کرد و مطابق با آن واکنش نشان داد.

پس در مثالی که مطرح شد، باید برای گره خصوصیت شماره موبایل هر فرد، ابتدا نوع حساب کاربری آن فرد بررسی شود و اگر شرایط موردنیاز فراهم بود،‌شماره برگردانده شود و درغیراین‌صورت،‌ خطای عدم دسترسی\footnote{\lr{Access Denied}} اعلام شود. این عملیات باید در تابع رافع\footnote{\lr{Resolver}} این خصوصیت تعریف شود و بنابراین، گرف‌کیوال دیگر از تابع رافع بدیهی\footnote{\lr{Trivial}} برای این خصوصیت (شماره موبایل) استفاده نخواهد کرد.

در توضیح محتویات شی \lr{Context} باید به شیوه احرازهویت استفاده شده اشاره شود. برای انجام احرازهویت در این سامانه از بلیت‌های \lr{JWT} استفاده شده است. این بلیت‌ها در خود سامانه اصلی و درهنگام تکمیل شدن احرازهویت دومرحله‌ای (که در فصل پیش به آن پرداختیم) ساخته شده و توسط یک زوج کلید نامتقارن 256 بیتی رمزنگاری می‌شود؛ البته این رمزنگاری برای حفظ اصالت بلیت تولیدشده است و نه حفظ محرمانگی. این بلیت در هر درخواست کاربر که احرازهویت نیاز داشته باشد باید تحت سرآیند اچ‌تی‌تی‌پی با مقدار کلید \lr{Authorization} و با مقدار \lr{Bearer TOKEN} که در آن \lr{TOKEN} برابر با بلیت دریافتی قبلی می‌باشد ارسال شود\cite{jwt}. 

\subsection{پایگاه‌داده اصلی و آزمایشی}

با توجه به این‌که در این سامانه از آزمون‌های خودکار (بخصوص برای تولید داده‌های نمونه) استفاده شده، بنابراین باید پایگاه‌داده اصلی آن از پایگاه‌داده آن وقتی درحال اجرای آزمون‌های خودکار است جدا باشد. برای این‌کار راحت‌ترین راه استفاده از دو نام جداگانه برای پایگاه‌داده‌ها در دستور برقراری ارتباط است.\\

اما مشکل از جایی شروع می‌شود که سامانه اصلی برای اجرای خود باید در همان لحظه ابتدای اجرا به پایگاه‌داده متصل شود (زیرا مدل داده‌ها برای اجرا شدن نیاز به ارتباط اولیه با پایگاه‌داده دارد)، از طرفی آزمون‌های خودکار نیز بلافاصله بعد از اجرا شدن کل سامانه، اجرا می‌شوند.

این همزمانی باعث می‌شود بخش اصلی کد، متوجه این نباشد که این اجرای واقعی است و یا جهت آزمایش سامانه! بنابراین باید این بخش آزمون‌ها با سازوکاری به بخش اصلی بفهماند که این اجرا صرفا برای آزمایش است و بخش اصلی باید به پایگاه‌داده آزمایشی متصل شود. خود بخش آزمون نیز باید اجرای آزمون‌ها را به زمان برقراری کامل ارتباط پایگاه‌داده موکول کند.\\

برای حل این چالش، از ترکیب چند ساختار غیرهمزمان‌\footnote{\lr{Asynchronous}}سازی در جاواسکریپت به نام \lr{Promise}ها با یک متغیر محیطی\footnote{\lr{Environment Variable}} استفاده شده است.

\section{چالش‌های محدودیت چارچوب‌ها}

در این بخش به چالش‌هایی پرداخته شده‌است که ناشی از محدودیت موجود در چارچوب‌های مورد استفاده است. البته چالش‌های بیشتری نیز وجود داشته، ‌اما در این‌جا فقط به ذکر دو مورد مهم از آن‌ها اکتفا می‌شود.

\newpage

\subsection{مسئله‌ی \lr{N+1} گرف‌کیوال}

این مشکل وقتی اتفاق می‌افتد که در کوئری ارسالی به گرف‌کیوال، چند سطح از  داده درخواست شده باشد که\cite{gql:nplusone}:
\begin{enumerate}
	\item این سطوح هر کدام نیازمند ارتباط با پایگاه‌داده باشند، و
	\item سطح بیرونی، یک آرایه را برگرداند.
\end{enumerate}

در این‌صورت، 1 ارتباط با پایگاه‌داده برای سطح بیرونی و \lr{N} ارتباط نیز به تعداد اعضای آرایه بیرونی برای پر کردن داده سطح دوم تمام اعضای آرایه (که \lr{N} عضو دارد) موردنیاز خواهد بود. درحالی‌که اگر همین مدل را با \lr{REST} پیاده می‌کردیم، فقط به 2 کوئری نیاز داشتیم. یعنی بجای کل \lr{N} کوئری در سطح داخلی، یک کوئری دیگر نیاز بود\cite{gql:nplusone}.

این مسئله، به نوعی جزو مسائل باز حساب می‌شود. دلیل این مسئله، اشتباه بودن گره‌های ایجاد شده برای گرف‌کیوال، و یا استفاده از پایگاه‌های داده \lr{SQL} است! درواقع وقتی از این نوع پایگاه‌های داده برای گرف‌کیوال استفاده شود، احتمال رخداد چنین چالشی بسیار بیشتر از زمانی است که از پایگاه‌داده مونگو استفاده می‌شود.\\

راه‌حل این مشکل زمانی‌که از اس‌کیوال‌ها استفاده شود، تجمیع کوئری‌های پایگاه‌داده از درون سطوح داخلی گره‌های گرف‌کیوال بهمراه سطوح خارجی، و اجرای یکباره‌ی آن‌هاست. برای این‌کار می‌توان از کتابخانه دیتالودر\footnote{\lr{ِDataloader}} که توسط خود سازنده گرف‌کیوال طراحی و پیاده‌سازی شده است استفاده کرد.

اما در این پروژه، باتوجه به این‌که از پایگاه‌داده مونگو استفاده شده، سعی شده که در تمام نقاط و گره‌های گرف‌کیوال، داده‌های موردنیاز در همان سطح بیرونی از مونگو گرفته شود و در سطوح داخلی، صرفا خصوصیات آن اشیا انتخاب شده و برگردانده شوند و به لطف مدل داده‌ی تودرتوی این پروژه، نیاز به پرس‌وجوی مجدد از پایگاه‌داده در سطوح داخلی گرف‌کیوال وجود نداشته باشد.

درواقع و از دیدگاه پیاده‌سازی، دو گره اصلی \lr{me} و \lr{baseUser} نقطه اصلی اتصال گرف‌کیوال به پایگاه‌داده است و در این دو گره، شی کاربر موردنظر بطور کامل از پایگاه‌داده دریافت می‌شود و بجز برای اعمال تغیرات، دسترسی دیگری به پایگاه‌داده در سطوح داخلی‌تر ایجاد نخواهد شد و پس از آن، صرفا از خصوصیات متغیرهای موجود درحافظه استفاده می‌شود. 

\newpage

\subsection{مرتب‌سازی طبق فاصله مکانی در مونگو}

از مهم‌ترین ویژگی‌های این سامانه، کار آن با موقعیت‌های مکانی و مختصات‌های جغرافیایی است. یکی از دلایل انتخاب پایگاه‌داده مونگو برای این پروژه نیز همین بوده است؛ زیرا مونگو امکانات وسیعی را در حوزه کار با مختصات جغرافیایی بطور پیشفرض فراهم می‌کند. برای مثال، تمامی اشیا مختصات درون یک مجموعه\footnote{\lr{Collection}} را فهرست\footnote{\lr{Index}} می‌کند تا در پرس‌وجوها، بسیار سریع به آن‌ها دست پیدا کند.

قابلیت اخیر ذکر شده، امکان جستجوی اسناد\footnote{\lr{Documents}}ی که درآن‌ها مختصات نزدیک‌تری به یک نقطه خاص وجود دارد در مدت زمان بسیار کوتاهی فراهم می‌کند. حتی اگر مجموعه اسناد بسیار زیاد و در مقیاس چند ده هزار عدد باشند، باز هم جستجو به لطف عملیات فهرست که روی مختصات در هنگام اضافه شدن آن‌ها به مجموعه انجام شده است، در مدت زمانی با مقیاس میلی‌ثانیه انجام خواهد شد.\\

در این‌جا لازم است مفاهیم استفاده شده را برای فهم بهتر مسئله به معادل اس‌کیوال آن نظیر کنیم. مجموعه در مونگو،‌ همان جدول و سند، همان ردیف جداول در اس‌کیوال هستند. اشیا و کلیدهای داخلی یک سند در مونگو، همان ستون‌ها و درمقیاس یک ردیف، معادل همان سلول‌های یک جدول در اس‌کیوال می‌باشند.

همان‌طور که در فصل دوم و در توضیح مفاهیم پایه مونگو گفته شد، این پایگاه‌داده از نوعی دستورات به نام دستورات تجمیع\footnote{\lr{Aggregation}} استفاده می‌کند. این دستورات در واقع مجموعه‌ای از مراحل تجمیع را درون خود دارند که در حین اجرای آن، مونگو کل داده‌های درون مجموعه را بعنوان ورودی اولین مرحله وارد آن می‌کند و خروجی مرحله اول تولید می‌شود؛ سپس خروجی تولید شده بعنوان ورودی وارد مرحله دوم می‌شود و همین روال تا مرحله آخر ادامه پیدا می‌کند. خروجی تولید شده از مرحله آخر،‌ بعنوان خروجی کل دستور تجمیع شناخته شده و بازگردانده می‌شود. درواقع این مراحل یک نوع خط لوله\footnote{\lr{Pipeline}} هستند و با نام مراحل خط لوله تجمیع در مونگو شناخته می‌شوند.

\newpage

دستورات مختلفی بعنوان مراحل خط لوله تجمیع در مونگو وجود دارند. اما دستور مهمی که چالش پیشِ‌رو را ایجاد کرده است،‌ دستور خط لوله \lr{\$GeoNear} است. این دستور بر روی یک کلید دارای شی مختصات از هر سند اعمال می‌شود و اسناد خروجی را به ترتیب فاصله (از نزدیک به دور) از یک نقطه ثابت داده شده مرتب می‌کند. این دستور خط لوله یک محدودیت بسیار مهم دارد و آن، اجبار مونگو در اولین مرحله بودن این دستور از خط لوله تجمیع است. پس برای استفاده از این دستور حتما باید آن را در اولین مراحل از خط لوله تجمیع اجرایی قرار داد\cite{mongo:geo}.\\

مشکل بعدی وقتی ایجاد می‌شود که درون یک سند از مجموعه، آرایه‌ی مستقیم یا غیرمستقیمی از اشیا مختصات داشته باشیم. در مدل داده‌های این سامانه، هر کاربر یک سند داده خواهد داشت. بنابراین اگر آن کاربر فروشنده باشد، آرایه‌ای از فروشگاه‌ها خواهد داشت و هر فروشگاه نیز یک مختصات و موقعیت مکانی مخصوص به خود دارد. پس در این‌جا نیز هر سند فروشنده از مجموعه اصلی سامانه، شامل آرایه‌ای از مختصات‌ها خواهد بود. مشکل اصلی این است که خروجی دستور خط لوله \lr{\$GeoNear} به ازای هر سند، یک سند خواهد بود؛ حتی اگر آن سند دارای آرایه‌ای از مختصات باشد. پس معلوم نمی‌شود که برای یک فروشنده، کدام فروشگاه به خریدار نزدیک‌تر بوده و فقط کل فروشگاه‌های آن فروشنده بازگردانده می‌شود.\\

البته دستور \lr{\$GeoNear} در نسخه‌های آخر مونگو، کلیدی را درخروجی اضافه کرده که صریحا مشخص می‌کند درصورت آرایه بودن مختصات‌ها، کدام یکی از آن‌ها نزدیک‌ترین به نقطه ثابت داده شده بوده است، اما ترتیب مختصات دیگر را (بجز نزدیک‌ترین مختصات درون آرایه‌های مختصات همان یک سند) اصلاح نمی‌کند\cite{mongo:geo}؛ این مشکل اول است.

چالش اخیر ذکر شده را براحتی می‌توان درون سامانه اصلی (بیرون از دستورات بومی پایگاه‌داده) و یا حتی در نرم‌افزار کاربردی در سمت کاربر رفع کرد. زیرا هر فروشنده تعداد زیادی فروشگاه نخواهد داشت و زمان محاسبه موردنیاز برای مرتب کردن فروشگاه‌های یک فروشنده زیاد نخواهد بود و در مقیاس میلی‌ثانیه انجام خواهد شد.

\newpage

اما این تنها مشکل و چالش نیست؛ ما مطمئن نیستیم که اگر فروشگاه‌های هر فروشنده را جداگانه مرتب کنیم، و آن‌گاه خود فروشنده‌ها را طبق فروشگاه‌های اول هر فروشنده (که نزدیک‌ترین از هرکدام خواهد بود) مرتب نماییم، خروجی ایجاد شده کاملا صحیح باشد! زیرا ممکن است فروشگاه دوم  فروشنده دوم، از فروشگاه دوم فروشنده اول،‌ به نقطه ثابت داده شده (که موقعیت خریدار است) نزدیک‌تر باشد! هرچند این اتفاق به ندرت ممکن است رخ دهد و بخصوص وقتی خروجی درون یک لیست بزرگ نمایش داده شود، شاید اشتباه بودن یک ترتیب در عناصر داخلی آن لیست برای کاربری که درحال مشاهده تمام آن‌ها در همان لحظه و بصورت یکجا است چندان اهمیتی نداشته باشد، اما بهرحال اگر بصورت تئوری به مسئله نگاه شود، این موضوع یک نقص و چالش است.

برای حل این چالش، بهترین راهی که بنظر می‌رسد، باز و پخش کردن فروشگاه‌ها از آرایه‌های فروشنده‌ها و همسطح کردن تمام آن‌ها از طریق دستور مرحله خط لوله تجمیع \lr{\$Unwind} است. اما آن محدودیت اولیه که از دستور \lr{\$GeoNear} گفته شد، این‌جا مشکل‌ساز می‌شود! طبق این محدودیت مونگو،‌ نمی‌توان از یک مرحله \lr{\$Unwind} پیش از مرحله \lr{\$GeoNear} استفاده کرد. پس عملا این راه‌حل ناممکن خواهد بود!\cite{mongo:geostack}\\

درحال حاضر و در این پروژه، راه‌حلی برای این شرایط بسیار خاص پیاده‌سازی نشده؛ زیرا همان‌طور که توضیح‌داده شد از اهمیت بالایی در کاربرد برخوردار نیست، اما در این‌جا بصورت عملیاتی روش‌های حل آن توضیح داده خواهند شد.

راه‌حل‌هایی وجود دارند که بسیار پرهزینه هستند؛ مثل اصلاح نقوص احتمالی حاصل از جستجو و مرتب‌سازی بدون درنظرگرفتن این مشکل، در سامانه اصلی و یا در نرم‌افزار کاربردی. چون تعداد کل فروشگاه‌ها بسیار بالاست این عملیات قطعا از پیچیدگی زمانی بالایی برخوردار خواهد بود و گلوگاه\footnote{\lr{Bottleneck}} محاسبات مرتب‌سازی این سامانه خواهد شد. این روش هم منطقی نیست.

بهترین راهی که برای حل این مسئله در تحقیقات و بررسی‌های انجام شده یافت شد، ایجاد یک مجموعه (معادل جدول) موقت از خروجی‌های مرتب‌سازی عادی در این مرحله (بدون توجه به این چالش)، و سپس استفاده مجدد از یک \lr{\$GeoNear} دیگر در مجموعه موقت ایجاد شده و درنهایت حذف مجموعه موقت است. همچنین می‌توان مدل‌داده‌ها را تغییر داد و فروشگاه‌ها را بصورت یک مجموعه جداگانه ارائه کرد (که البته هزینه این تغییر ممکن است بیشتر باشد!).



% \chapter{خروجی و آزمون‌ها}

در این فصل سعی شده است تا با ارائه تعدادی تصاویر از محیط سامانه و عملکرد آن، خروجی سامانه و همچنین برخی از آزمون‌های تولید شده نیز ارائه شود. البته تعداد بسیار بیشتری تصویر و بخش قابل ارائه بود اما در این‌صورت حجم این گزارش بسیار زیاد می‌شد. همچنین  بدلیل این‌که جزئیات کار سامانه در فصول قبلی شرح داده شده از توضیح تصاویر فوق صرف‌نظر شده است.

\section{سامانه اصلی}

\subsection{ساختار فایل‌ها}

\begin{figure}[H]
	\centering
	\includegraphics[scale=0.5]{files}
	\caption{فایل‌های سامانه اصلی}
	\label{fig:files}
\end{figure}

\subsection{خروجی سامانه اصلی}

\begin{figure}[H]
	\centering
	\includegraphics[scale=0.45]{backend}
	\caption{درخواست ازمایشی دریافت پیشنهاد مبتنی بر مکان}
	\label{fig:backend}
\end{figure}


\newpage

\subsection{خروجی نرم‌افزار کاربردی}

\begin{figure}[H]
	\centering
	\includegraphics[scale=0.1]{1}
	\includegraphics[scale=0.1]{2}
	\includegraphics[scale=0.1]{3}
	\caption{به ترتیب از راست، لیست صفحه اصلی، لیست محصولات یک فروشگاه، صفحه یک محصول}
	\label{fig:app1}
\end{figure}

\begin{figure}[H]
	\centering
	\includegraphics[scale=0.1]{4}
	\includegraphics[scale=0.1]{5}
	\includegraphics[scale=0.1]{6}
	\caption{به ترتیب از راست، صفحه افزودن فروشگاه، صفحه تغییر یک محصول، صفحه مشاهده و آمارگیری لیست کوپن‌ها}
	\label{fig:app2}
\end{figure}

\section{آزمون سامانه}

\begin{figure}[H]
	\centering
	\includegraphics[scale=0.3]{test1}
	\caption{آزمون واحد تابع خطی‌ساز اشیا}
	\label{fig:test1}
\end{figure}


\begin{figure}[H]
	\centering
	\includegraphics[scale=0.4]{test2}
	\caption{آزمون واحد تولید بلیت با \lr{JWT}}
	\label{fig:test2}
\end{figure}


\begin{figure}[H]
	\centering
	\includegraphics[scale=0.3]{test3}
	\caption{آزمون واحد توابع کمکی کار با آرایه \lr{ArrayFilter}}
	\label{fig:test3}
\end{figure}




% \chapter{پیشنهادات و جمع‌بندی}

\section{کارهای آینده}

از مهم‌ترین اقداماتی که برای بهبود این پروژه می‌توان انجام داد، حل چالش بوجود آمده در زمینه پرس‌وجوهای مکانی مونگو است. به این‌صورت که می‌توان روشی که فصل 5 پیشنهاد شد را پیاده‌سازی کرد و تاثیر آن را بر عملکرد سامانه اندازه گرفت. زیرا همان‌طور که توضیح داده شده بود، این چالش در عمل چندان مشکل‌ساز نخواهد بود، اما درصورتی‌که اصلاح آن با روش پیشنهاد شده تاثیر مخرب ناچیزی بر عملکرد سامانه داشته باشد، اصلاح و غلبه بر آن مفید خواهد بود.\\

علاوه بر این، همانطور که تاکید شد، وظیفه اصلی این پروژه ارائه پیشنهاداتی از محصولات فروشگاه‌ها به کاربرانی است که یا از نظر موقعیت مکانی، احتمال گذر آن‌ها از آن فروشگاه‌ها بالاست و یا از پیش، علاقه‌مندی آن‌ها به آن محصولات مشخص شده است. موارد دیگری نیز می‌توان بعنوان کارهای فراتر از این پروژه انجام داد؛ برای مثال،

\begin{enumerate}

	\item می‌توان با تحلیل موقعیت مکانی هر کاربر در طول روز، مکان‌هایی که احتمالا فروشگاه هستند، ولی در این سامانه هنوز ثبت نشده‌اند را بعنوان نقطه مورد علاقه  در سامانه ثبت کرد تا پس از بررسی پشتیبان سامانه، بعنوان فروشگاه اضافه شود.
	\item پس از توافق با شرکت‌های ارائه دهنده خدمات تاکسی اینترنتی، می‌توان برای مسافرین این تاکسی‌ها درطول مسیر، پیشنهاداتی را ارائه داد که در صورت توقف برای خرید از فروشگاه پیشنهاد شده، هزینه سفر آن‌ها بصورت رایگان و از طرف این سامانه پرداخت شود.
	\item علاوه بر پیشنهاد مبتنی بر موقعیت مکانی و محصولات مشابه موردعلاقه کاربر، می‌توان پیشنهادات را از زنجیره‌هایی از مشتریانی که به فروشگاه‌های مشابهی مراجعه کرده‌اند نیز بدست آورد. در اینصورت، این سامانه عملا از ویژگی‌های یک شبکه اجتماعی مبتنی بر مکان برخوردار خواهد شد.
	
\end{enumerate}

\newpage

\section{جمع‌بندی}

امروزه بسیاری از سامانه‌های تجاری، اقدام به ارائه پیشنهادات به کاربران خود مطابق با سلیقه آن‌ها می‌کنند. اگر خدمات ارائه شده حضوری باشند و در ارائه این پیشنهادات، موقعیت مکانی کاربران بعنوان یک شاخص مهم مورد توجه قرار گیرد، می‌تواند مشتریان قابل توجهی به خود جذب نماید. سامانه تولید شده در این پروژه شامل اجزای مختلفی از جمله زیرساخت سامانه، نرم‌افزار کاربردی تلفن همراه هوشمند و پنل تحت وب برای مدیران است که باید در تعامل با یکدیگر، اهداف اصلی تعیین شده برای این پروژه را محقق نمایند. خدمت ارائه شده از طریق این نرم‌افزار، نمایش اطلاعات محصولات و فروشگاه‌های متنوع به کاربران علاقه‌مند و ارائه کوپن‌های تخفیف به آن‌هاست. درفرآیند تولید این سکوی نرم‌افزاری، چالش‌های مختلفی اعم از چالش‌های فرآیندی (از جمله نحوه دسته‌بندی وظایف)،‌ الگوریتمی (مانند نحوه محاسبه علاقه‌مندی‌ها در پایگاه‌داده استفاده شده) و معماری (مانند نحوه انتقال داده‌های محلی در طول نرم‌افزار تلفن همراه هوشمند کاربر) وجود داشته که سعی شده تا حد امکان با راه‌حل‌های مناسبی رفع شده و مسیر توسعه این سامانه هموار شود. کار انجام شده‌ی پیشِ‌رو، نقطه پایانی برای این پروژه نبوده و می‌توان ایده‌های زیادی برای پیشرفت و مسیر آینده این پروژه متصوّر شد؛ از جمله مواردی که در بخش قبلی توضیح داده شد.\\

کدهای پیاده‌سازی شده این سامانه در سه منبع از سامانه کنترل نسخه گیت‌هاب در آدرس‌های زیر موجود است:

\begin{enumerate}
	\item کد سامانه اصلی: \url{https://github.com/mmkhmmkh/LBSR_backend}
	\item کد نرم‌افزار کاربردی تلفن‌های هوشمند: \url{https://github.com/mmkhmmkh/LBSR_rn}
	\item کد پنل مدیران: \url{https://github.com/mmkhmmkh/LBSR_admin}
\end{enumerate}



%--------------------------------------------------------------------------appendix( مراجع و پیوست ها)
\chapterfont{\vspace*{-2em}\centering\LARGE}%

\appendix
\bibliographystyle{plain-fa}
\bibliography{references}
%\chapter*{‌پیوست}
\markboth{پیوست}{}
\addcontentsline{toc}{chapter}{پیوست}
موضوعات مرتبط با متن گزارش پایان نامه كه در يكی از گروه‌های زير قرار می‌گيرد، در بخش پيوست‌ها آورده شوند:
\begin{enumerate}
\item  اثبات های رياضی يا عمليات رياضی طولانی‌.‌
\item داده و اطلاعات نمونه (های) مورد مطالعه (\lr{Case Study}) چنانچه طولانی باشد‌.‌
\item نتايج كارهای ديگران چنانچه نياز به تفصيل باشد‌.‌
\item مجموعه تعاريف متغيرها و پارامترها، چنانچه طولانی بوده و در متن به انجام نرسيده باشد‌.‌
\end{enumerate}
% براي شماره‌گذاري روابط، جداول و اشكال موجود در پيوست‌ از ساختار متفاوتي نسبت به متن اصلي استفاده مي‌شود كه در زير به‌عنوان نمونه نمايش داده شده‌است. 
% \begin{equation}
%F=ma
%\end{equation}
\section*{کد میپل }
\begin{latin}
\begin{verbatim}

with(DifferentialGeometry):
with(Tensor):
DGsetup([x, y, z], M)
																	frame name: M
a := evalDG(D_x)
																	D_x
b := evalDG(-2 y z D_x+2 x D_y/z^3-D_z/z^2)


\end{verbatim}
\end{latin}
%--------------------------------------------------------------------------dictionary(واژه نامه ها)
%اگر مایل به داشتن صفحه واژه‌نامه نیستید، خط زیر را غیر فعال کنید.
\parindent=0pt
%%
\chapter*{واژه‌نامه‌ی فارسی به انگلیسی}
\pagestyle{style9}

\addcontentsline{toc}{chapter}{واژه‌نامه‌ی فارسی به انگلیسی}
%%%%%%
\begin{multicols*}{2}

{\bf آ}
\vspace*{3mm}


\farsiTOenglish{اسکالر}{Scalar}


\vspace*{3mm}
{\bf ب}
\vspace*{3mm}

\farsiTOenglish{بالابر}{Lift}


\vspace*{3mm}
{\bf پ}
%%\vspace*{3mm}

\farsiTOenglish{پایا}{Invariant}



\vspace*{3mm}
{\bf ت}
%%\vspace*{3mm}

\farsiTOenglish{ تناظر }{Correspondence}


\vspace*{3mm}
{\bf ث}
%%\vspace*{3mm}

\farsiTOenglish{ثابت‌ساز}{Stabilizer}

\vspace*{3mm}
{\bf ج}
%%\vspace*{3mm}

\farsiTOenglish{جایگشت}{Permutation}



\vspace*{3mm}
{\bf چ}
%%\vspace*{3mm}


\farsiTOenglish{چند جمله‌ای }{Polynomial}

\vspace*{3mm}
{\bf ح}
%%\vspace*{3mm}

\farsiTOenglish{حاصل‌ضرب دکارتی}{Cartesian product}


\vspace*{3mm}
{\bf خ}
%%\vspace*{3mm}

\farsiTOenglish{خودریختی}{Automorphism}

\vspace*{3mm}
{\bf د}
%%\vspace*{3mm}

\farsiTOenglish{درجه}{Degree}


\vspace*{3mm}
{\bf ر}
%%\vspace*{3mm}


\farsiTOenglish{ریزپردازنده}{microprocessor}


\vspace*{3mm}
{\bf ز}
%%\vspace*{3mm}


\farsiTOenglish{زیرمدول}{Submodule}


\vspace*{3mm}
{\bf س}
%%\vspace*{3mm}

\farsiTOenglish{سرشت}{Character}


\vspace*{3mm}
{\bf ص}
%%\vspace*{3mm}

\farsiTOenglish{صادقانه}{Faithful}

\vspace*{3mm}
{\bf ض}
%%\vspace*{3mm}

\farsiTOenglish{ضرب داخلی}{Inner product}

\vspace*{3mm}
{\bf ط}
%%\vspace*{3mm}


\farsiTOenglish{طوقه}{Loop}


\vspace*{3mm}
{\bf ظ}
%%\vspace*{3mm}


\farsiTOenglish{ظرفیت}{Valency}
 
\vspace*{3mm}
{\bf ع}
%%\vspace*{3mm}


\farsiTOenglish{عدم مجاورت}{Nonadjacency}



\vspace*{3mm}
{\bf ف}
%%\vspace*{3mm}

\farsiTOenglish{فضای برداری}{Vector space}



\vspace*{3mm}
{\bf ک}
%%\vspace*{3mm}

\farsiTOenglish{کاملاً تحویل‌پذیر}{Complete reducibility}


\vspace*{3mm}
{\bf گ}
%%\vspace*{3mm}


\farsiTOenglish{گراف}{Graph}



\vspace*{3mm}
{\bf م}
%%\vspace*{3mm}

\farsiTOenglish{ماتریس جایگشتی}{Permutation matrix }


\vspace*{3mm}
{\bf ن}
%%\vspace*{3mm}

\farsiTOenglish{ناهمبند}{Disconnected}


\vspace*{3mm}
{\bf و}
%%\vspace*{3mm}

\farsiTOenglish{وارون‌پذیر}{Invertible}


\vspace*{3mm}
{\bf ه}
%%\vspace*{3mm}

\farsiTOenglish{همبند}{Connected}



\vspace*{3mm}
{\bf ی}
%%\vspace*{3mm}

\farsiTOenglish{یال}{Edge}




\end{multicols*}%
%%%%%%%
\chapter*{ واژه‌نامه‌ی انگلیسی به فارسی}
\pagestyle{style9}
\lhead{\thepage}\rhead{واژه‌نامه‌ی انگلیسی به فارسی}
\addcontentsline{toc}{chapter}{واژه‌نامه‌ی انگلیسی به فارسی}

\LTRmulticolcolumns
\begin{multicols}{2}
{\hfill\bf  \lr{A}}
%%\vspace*{1.5mm}

\englishTOfarsi{Automorphism}{خودریختی}

\vspace*{3mm}
{\hfill\bf   \lr{B}}
%%\vspace*{1.5mm}

\englishTOfarsi{Bijection}{دوسویی}

\vspace*{3mm}
{\hfill\bf   \lr{C}}
%%\vspace*{1.5mm}

\englishTOfarsi{Cycle group}{گروه دوری}

\vspace*{3mm}
{\hfill\bf   \lr{D}}
%%\vspace*{1.5mm}

\englishTOfarsi{Degree}{درجه}

\vspace*{3mm}
{\hfill\bf   \lr{E}}
%%\vspace*{1.5mm}

\englishTOfarsi{Edge}{یال}

\vspace*{3mm}
{\hfill\bf   \lr{F}}
%%\vspace*{1.5mm}

\englishTOfarsi{Function}{تابع}

\vspace*{3mm}
{\hfill\bf   \lr{G}}
%%\vspace*{1.5mm}

\englishTOfarsi{Group}{گروه}

\vspace*{3mm}
{\hfill\bf   \lr{H}}
%%\vspace*{1.5mm}

\englishTOfarsi{Homomorphism}{همریختی}

\vspace*{3mm}
{\hfill\bf   \lr{I}}
%%\vspace*{1.5mm}

\englishTOfarsi{Invariant}{پایا}

\vspace*{3mm}
{\hfill\bf   \lr{L}}
%%\vspace*{1.5mm}

\englishTOfarsi{Lift}{بالابر}

\vspace*{3mm}
{\hfill\bf   \lr{M}}
%%\vspace*{1.5mm}

\englishTOfarsi{Module}{مدول}

\vspace*{3mm}
{\hfill\bf   \lr{N}}
%%\vspace*{1.5mm}

\englishTOfarsi{Natural map}{نگاشت طبیعی}

\vspace*{3mm}
{\hfill\bf   \lr{O}}
%%\vspace*{1.5mm}

\englishTOfarsi{One to One}{یک به یک}

\vspace*{3mm}
{\hfill\bf   \lr{P}}
%%\vspace*{1.5mm}

\englishTOfarsi{Permutation group}{گروه جایگشتی}

\vspace*{3mm}
{\hfill\bf   \lr{Q}}
%%\vspace*{1.5mm}

\englishTOfarsi{Quotient graph}{گراف خارج‌قسمتی}

 \vspace*{3mm}
{\hfill\bf   \lr{R}}
%%\vspace*{1.5mm}

\englishTOfarsi{Reducible}{تحویل پذیر}

\vspace*{3mm}
{\hfill\bf   \lr{S}}
%%\vspace*{1.5mm}

\englishTOfarsi{Sequence}{دنباله}

 \vspace*{3mm}
{\hfill\bf   \lr{T}}
%%\vspace*{1.5mm}

\englishTOfarsi{Trivial character}{سرشت بدیهی}

\vspace*{3mm}
{\hfill\bf   \lr{U}}
%%\vspace*{1.5mm}

\englishTOfarsi{Unique}{منحصربفرد}

\vspace*{3mm}
{\hfill\bf   \lr{V}}
%%\vspace*{1.5mm}

\englishTOfarsi{Vector space}{فضای برداری}
\end{multicols}
%--------------------------------------------------------------------------index(نمایه)
%اگر مایل به داشتن صفحه نمایه نیستید، خط زیر را غیر فعال کنید.
\pagestyle{style7}
\printindex
\pagestyle{style7}
%%کلمات کلیدی انگلیسی
\latinkeywords{Write a 3 to 5 KeyWords is essential. Example: AUT, M.Sc., Ph. D,..}
%چکیده انگلیسی

\en-abstract{
This page is accurate translation from Persian abstract into English.
}
%%%%%%%%%%%%%%%%%%%%% کدهای زیر را تغییر ندهید.

\newpage
\thispagestyle{empty}
\begin{latin}
\section*{\LARGE\centering Abstract}

\een-abstract

\vspace*{.5cm}
{\large\textbf{Key Words:}}\par
\vspace*{.5cm}
\elatinkeywords
\end{latin}
% در این فایل، عنوان پایان‌نامه، مشخصات خود و چکیده پایان‌نامه را به انگلیسی، وارد کنید.
%%%%%%%%%%%%%%%%%%%%%%%%%%%%%%%%%%%%
\baselineskip=.6cm
\begin{latin}

\latinfaculty{Department of ...}


\latintitle{Title of Thesis}


\firstlatinsupervisor{Dr. }

%\secondlatinsupervisor{Second Supervisor}

\firstlatinadvisor{Dr. }

%\secondlatinadvisor{Second Advisor}

\latinname{Name}

\latinsurname{Surname}

\latinthesisdate{Month \& Year}

\latinvtitle
\end{latin}

\end{document}